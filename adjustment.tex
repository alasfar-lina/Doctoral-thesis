\KOMAoptions{numbers=noenddot}
\usepackage{amsmath,amssymb,amsfonts,amsthm,epigraph,scrpage2}
\usepackage[ngerman,english]{babel}
\usepackage{centernot} 
\usepackage{multirow,mathtools}
\usepackage{pdfpages}
\usepackage{currvita,geometry}
\renewcommand*{\cvheadingfont}{\large\bfseries} % CV Titel
\renewcommand*{\cvlistheadingfont}{\bfseries\sffamily} % sub-�berschriften
\renewcommand*{\cvlabelfont}{\sffamily} % items
\usepackage[compress,numbers,sort]{natbib}
\usepackage[mathscr]{euscript}
\usepackage{placeins}
\usepackage{booktabs}
\usepackage{ragged2e}
\usepackage{etoolbox}
\usepackage{multirow}
\usepackage{amssymb}
\usepackage{bbold}
\usepackage{lieart}
\usepackage{booktabs,colortbl}
\usepackage{slashbox}
\usepackage[normalem]{ulem}
\usepackage{mathrsfs}
\usepackage{slashed,bbold}
%\usepackage{pict2e}
\usepackage{url}
\usepackage{multirow}
\usepackage{siunitx}
\usepackage{xcolor}
\usepackage{epigraph} 
\usepackage{rotfloat}
\usepackage[most]{tcolorbox}
%%%%%%%%%%%%%%%%%%%
% Colour Scheme
\definecolor{myblue}{HTML}{003366}
\usepackage{pgfplotstable}
\pgfplotsset{compat=1.8}
% see https://tex.stackexchange.com/a/60761/5001 for source of \breakingcomma
% (requires the "breqn" package; that's why it's loaded above)
\definecolor{rulecolor}{HTML}{003366}
\definecolor{tableheadcolor}{gray}{0.92}
% Following is taken from Werner: http://tex.stackexchange.com/a/33761/3061
% and modified for my needs
%
% Command \topline consists of a (slightly modified)
% \toprule followed by a \heavyrule rule of colour tableheadcolor
% (hence, 2 separate rules)
\newcommand{\toplinetwo}{ %
	\arrayrulecolor{rulecolor}\specialrule{0.1em}{\abovetopsep}{0pt}%
	\arrayrulecolor{white}\specialrule{\belowrulesep}{0pt}{0pt}%
	\arrayrulecolor{rulecolor}}

\newcommand{\topline}{ %
	\arrayrulecolor{rulecolor}\specialrule{0.1em}{\abovetopsep}{0pt}%
	\arrayrulecolor{tableheadcolor}\specialrule{\belowrulesep}{0pt}{0pt}%
	\arrayrulecolor{rulecolor}}
% Command \midline consists of 3 rules (top colour tableheadcolor, middle colour black, bottom colour white)
\newcommand{\midtopline}{ %
	\arrayrulecolor{tableheadcolor}\specialrule{\aboverulesep}{0pt}{0pt}%
	\arrayrulecolor{rulecolor}\specialrule{\lightrulewidth}{0pt}{0pt}%
	\arrayrulecolor{white}\specialrule{\belowrulesep}{0pt}{0pt}%
	\arrayrulecolor{rulecolor}}
% Command \bottomline consists of 2 rules (top colour
\newcommand{\bottomline}{ %
	\arrayrulecolor{white}\specialrule{\aboverulesep}{0pt}{0pt}%
	\arrayrulecolor{rulecolor} %
	\specialrule{\heavyrulewidth}{0pt}{\belowbottomsep}}%


\newcommand{\midheader}[2]{%
	\midrule\topmidheader{#1}{#2}}
\newcommand\topmidheader[2]{\multicolumn{#1}{c}{\textsc{#2}}\\%
	\addlinespace[0.5ex]}
%%%%%%%%%%%%%%%%%%%%%%%%%
\definecolor{Mahogany}{rgb}{0.36,0.54,0.66}
\definecolor{Cayenne}{rgb}{0.502,0.0,0.0}
\definecolor{Steel}{rgb}{0.4,0.4,0.4}
\definecolor{Tri_blue}{rgb}{0.039,0.5098,0.8}
\definecolor{Tri_yellow}{rgb}{0.5529,0.5451,0.0549}
\newcommand{\la}[1]{\textcolor{Cayenne}{#1}}
\definecolor{Gray}{gray}{0.95}
\newcommand{\CG}{\cellcolor{Gray}}
%\setcounter{secnumdepth}{3} % sub subsections numbering
%\setcounter{tocdepth}{3} % subsubsections inTOC

\usepackage[format=plain,singlelinecheck=false, font={sf,small},labelfont={bf,color=Steel}]{caption}
\DeclareCaptionLabelSeparator{cayenne_period}{\textcolor{Cayenne}{.} }
\captionsetup{labelsep=cayenne_period}

% Colors
\addtokomafont{chapter}{\color{Steel}}
\addtokomafont{section}{\color{Steel}}
\addtokomafont{subsection}{\color{Steel}}
\addtokomafont{subsubsection}{\color{Steel}}
\addtokomafont{paragraph}{\color{Steel}}
\addtokomafont{disposition}{\color{Steel}}
\addtokomafont{pagehead}{\color{Steel}}
\renewcommand{\pnumfont}{\color{Steel}} 
\addtokomafont{headsepline}{\color{Steel}} 
\pagestyle{scrheadings}

% Textcolor for chapters in TOC
\makeatletter
\let\stdl@chapter\l@chapter
\renewcommand*{\l@chapter}[2]{%
  \stdl@chapter{\textcolor{black}{#1}}{\textcolor{black}{#2}}}
\makeatother

%  labels in description environments
\renewcommand{\descriptionlabel}{\hspace\labelsep{}\sffamily\small\bfseries{}\color{Steel}{}}

%\makeatletter % dot after sections and all below
%\let\std@sect\@sect
%\def\@sect#1#2#3#4#5#6[#7]#8{\std@sect{#1}{#2}{#3}{#4}{#5}{#6}[#7.]{#8\color{Cayenne}{.}}}
%\makeatother
\usepackage{acronym}

\usepackage[leftcaption]{sidecap} % inner, outer,left,right
\sidecaptionvpos{figure}{t}

% Papiergr��e
%\setlength{\paperwidth}{21cm}
%\setlength{\paperheight}{25cm}
%\recalctypearea
%\usepackage[pass]{geometry}
%\usepackage[cross,a4,center]{crop}

%% Flattersatz
%\usepackage[document]{ragged2e} % Flattersatz
%\setlength{\RaggedRightParindent}{1em} % evtl. parskip


%% Sans Serif
%\usepackage{cmbright}
%\renewcommand{\familydefault}{\sfdefault}
%% Palatino
%\usepackage[sc]{mathpazo}
%\linespread{1.05}         % Palatino needs more leading (space between lines)
%\setkomafont{sectioning}{\normalcolor\bfseries} % Kapitel�berschriften

%%% Kapitel�berschriften: Mit gro�en Zahlen
%\usepackage{titlesec}
%\titleformat{\chapter}[display]
%{\bfseries\Large}
%{ %\Huge\textsc{\chaptertitlename} % f�r das Wort 'Kapitel'
%\hfill\fontsize{120}{70}\selectfont\color{lightgray}\textbf{\thechapter}}
%{-2ex}
%%{\filleft\fontsize{50}{70}\selectfont\scshape} % Kapit�lchen oder...
%{\filleft\fontsize{50}{70}\selectfont\textbf} % ...oder keine Kapit�lchen
%[\vspace{0ex}]
%
%%%% Part�berschriften
%\titleformat{\part}[display]
%{\bfseries\Large}
%{ %\Huge\textsc{\chaptertitlename} % f�r das Wort 'Kapitel'
%\hfill\fontsize{120}{70}\selectfont\color{lightgray}\textbf{\thepart}}
%{-2ex}
%{\filleft\fontsize{50}{70}\selectfont\scshape} % Kapit�lchen oder...
%%{\filleft\fontsize{50}{70}\selectfont\textbf} % ...oder keine Kapit�lchen
%[\vspace{0ex}]


\newcommand{\ER}{Erd\H{o}s-R\'enyi }
\newcommand{\BA}{Barab\'asi-Albert }
\newcommand{\mean}[1]{\left< #1 \right>}
\newcommand{\abs}[1]{\left| #1 \right|}
\newcommand{\norm}[1]{\lVert#1\rVert}
\newcommand{\mat}[1]{\mathbf{#1}}
\newcommand{\tgraph}{\mathcal{G}}

\theoremstyle{definition} % non-italic
\newtheorem{annahme}{Annahme} % braucht amsthm
\newtheorem{definition}{Definition}
\newtheorem{theorem}{Theorem}
\newtheorem{satz}{Satz}
\newtheorem{frage}{Frage}
%\input{watermarks/watermark.tex}
\DeclareMathOperator{\nnz}{nnz}

% + Graphicspath nach begin document

% aus Doi hyperref machen
%\newcommand*{\doi}[1]{\href{http://dx.doi.org/\detokenize{#1}}{doi: \detokenize{#1}}}



% Hyperlinks Setup
\hypersetup{
    unicode=false,          
    pdftoolbar=true,        
    pdfmenubar=true,        
    pdffitwindow=false,     
    pdfstartview={FitH},    
    pdftitle={Resurrecting bbh with kinematic shapes},
    pdfauthor={Lina Alasfar, Ramona Groeber, Christophe Grojean, Ayan Paul and Zhuoni Qian},
    pdfkeywords={Future Colliders} {dihiggs} {Machine Learning} {Shapley Values},
    pdfnewwindow=true,
    colorlinks=true,
    linkcolor=myblue,
    citecolor=myblue,
    filecolor=myblue,
    urlcolor=myblue,
    linktocpage=true
}
\newcommand{\SM}{\mathrm{SM}}
\newcommand{\NP}{\mathrm{NP}}
\newcommand{\hc}{\mathrm{h.c.}}
\newcommand{\MSbar}{\overline{\mathrm{MS}}}
%%%%%%%%%%%%%%%%%%%%%%%%%%%%%%%%%%%%%%%%
% autoref configuration
\renewcommand{\sectionautorefname}{section}
\renewcommand{\subsectionautorefname}{section}
\renewcommand{\appendixautorefname}{appendix}
\renewcommand{\tableautorefname}{table}
\renewcommand{\figureautorefname}{figure}
\def\equationautorefname~#1\null{Eq.\,(#1)\null}
\newcommand{\appendixref}[1]{\hyperref[#1]{appendix~\ref{#1}}}

%%%%%%def
%%%%%% def
\newcommand{\pt}{p_{\scriptscriptstyle T}}

\newcommand{\sssty}[1]{\scriptscriptstyle#1}

\newcommand{\be}{\begin{equation}}
	\newcommand{\ee}{\end{equation}}

\newcommand{\mt}{m_t}

\newcommand{\vev}[1]{\langle {#1} \rangle}
\newcommand{\lsim}{\lesssim}
\newcommand{\gsim}{\gtrsim}

\newcommand{\invab}{\si{\per \atto\barn}}
\newcommand{\femtobarn}{\si{\femto\barn}}
\newcommand{\nn}{\nonumber}
\newcommand{\perc}{\%}
\newcommand\sss{\scriptscriptstyle}

\newcommand{\gev}{\,\textrm{GeV}}
\newcommand{\mev}{\,\textrm{MeV}}
\newcommand{\pb}{\,\textrm{pb}}
\newcommand{\tev}{\,\textrm{TeV}}
\newcommand{\TO}{\rightarrow}


\newcommand{\gnote}[1]{\textbf{[G:} \textit{#1}\textbf{]}}
\newcommand{\rnote}[1]{\textbf{[R:}\textit{#1}\textbf{]}}
\newcommand{\ppgnote}[1]{\textbf{[PPG:}\textit{#1}\textbf{]}}
\newcommand{\tth}{t\bar{t}H}
\newcommand{\tthc}{\lambda_{\tth}}
\newcommand{\pptth}{pp\TO\tth}
\newcommand{\ggtth}{gg\TO\tth}
\newcommand{\qqtth}{q\bar{q}\TO\tth}
\newcommand{\qqtthg}{q\bar{q}\TO\tth g}
\newcommand{\qgtthq}{qg\TO\tth q}
\newcommand{\asa}[2]{\alpha_s^{#1}\alpha^{#2}}
\newcommand{\ord}[1]{\mathcal{O}{(#1)}}
\newcommand{\mh}{m_{ \sss h}}
\newcommand{\mz}{m_{ \sss Z}}
\newcommand{\dm}{\Delta_m}
\newcommand{\Gfer}{G_{ \sss F}}

%%%%%%%%%%%%%
%%%%%%%%%%%%%%%%%%%%%%%%%%%%% math %%%%%%%%%%%%%%%%%%%%%%%%%%%%%%%%
\def\vev#1{\left\langle #1\right\rangle}
\def\abs#1{\left| #1\right|}
\def\mod#1{\abs{#1}}
\def\Im{\mbox{Im}\,}
\def\Re{\mbox{Re}\,}
\def\Tr{\mbox{Tr}\,}
\def\det{\mbox{det}\,}
\def\etal{\hbox{\it et al.}}
\def\ie{\hbox{\it i.e.}{}}
\def\eg{\hbox{\it e.g.}{}}
\def\etc{\hbox{\it etc}{}}
%%%%%%%%%%%%%%%%%%%%%%%%%%%%%
\newcommand{\mr}[1]{\multirow{2}{*}{#1} }
\newcommand{\rb}[1]{\rotatebox{90}{#1}}
\newcommand{\neff}{n_{\rm eff}}
\renewcommand{\d}{{\rm d}}
\renewcommand{\bar}{\overline}
\newcommand{\hoppet}{\textsc{Hoppet}}
\newcommand{\R}[2]{$R_{#1#2}$}
%\newcommand{\as}{$\alpha_s$}
\newcommand{\rr}[1]{{\color{red}#1}}
\newcommand{\bb}[1]{{\color{blue}#1}}
\newcommand{\vs}{v_{\scriptscriptstyle S}}
\newcommand{\vh}{v_{\scriptscriptstyle H}}
\newcommand{\lambdah}{\lambda_{\scriptscriptstyle H}}
\newcommand{\muh}{\mu_{\scriptscriptstyle H}}
\newcommand{\LO}{\textrm{LO}}
\newcommand{\NLO}{\textrm{NLO}}
%%%%%%%%%%%%%

\DeclareOldFontCommand{\rm}{\normalfont\rmfamily}{\mathrm}
%\DeclareOldFontCommand{\sf}{\normalfont\sffamily}{\mathsf}
%\DeclareOldFontCommand{\tt}{\normalfont\ttfamily}{\mathtt}
\DeclareOldFontCommand{\bf}{\normalfont\bfseries}{\mathbf}
\DeclareOldFontCommand{\it}{\normalfont\itshape}{\mathit}
\DeclareOldFontCommand{\sl}{\normalfont\slshape}{\@nomath\sl}
\DeclareOldFontCommand{\sc}{\normalfont\scshape}{\@nomath\sc}
%\DeclareRobustCommand*\cal{\@fontswitch\relax\mathcal}
\DeclareRobustCommand*\mit{\@fontswitch\relax\mathnormal}

\def\abs#1{\left|#1\right|}

\newcommand{\bra}[1]{\mbox{$\langle\, #1 \mid$}}
\newcommand{\ket}[1]{\mbox{$\mid #1\,\rangle$}}
\newcommand\mydot{\!\cdot\!}
\newcommand\ep{\epsilon}
\newcommand\half{\frac{1}{2}}
\newcommand\quarter{\frac{1}{4}}
\newcommand\qb{\bar{q}}
\newcommand\ub{\bar{u}}
\newcommand\db{\bar{d}}
\newcommand\tb{\bar{t}}
\newcommand\sqs{\sqrt{s}}
\newcommand\epem{e^+e^-}
\newcommand\mpmm{\mu^+\mu^-}
\newcommand\irmv{\remove{i}{0.18}}
\newcommand\jrmv{\remove{j}{0.21}}
\newcommand\irmvb{\removeb{i}{0.18}{0.18}}
\newcommand\jrmvb{\removeb{j}{0.21}{0.17}}
\newcommand\isubrmv{\remove{i}{0.125}}
\newcommand\jsubrmv{\remove{j}{0.145}}
\newcommand\FKSpairs{{\cal P}_{\sss\rm FKS}}
\newcommand\FKSpairsred{\overline{{\cal P}}_{\sss\rm FKS}}
\newcommand\FKSelem{N_{\sss\rm FKS}}
\newcommand\FKSelemred{\overline{N}_{\sss\rm FKS}}
\newcommand\nchannels{N_{\rm ch}}
\newcommand\proc{\qqtth}
\newcommand\procB{r_{\sss B}}
\newcommand\procR{r_{\sss R}}
\newcommand\allproc{{\cal R}}
\newcommand\allprocnpo{\allproc_{n+1}}
\newcommand\allprocn{\allproc_{n}}
\newcommand\BornME{{\cal B}}
\newcommand\nini{n_{\sss I}}
\newcommand\nlight{n_{\sss L}}
\newcommand\nlightB{\nlight^{\sss (B)}}
\newcommand\nlightR{\nlight^{\sss (R)}}
\newcommand\nlightBorR{\nlight^{\sss (B/R)}}
\newcommand\nheavy{n_{\sss H}}
\newcommand\nzero{n_\emptyset}
\newcommand\ident{{\cal I}}
\newcommand\numofgr{N_d}
\newcommand\amp{{\cal A}}
\newcommand\ampmt{\amp^{(m,0)}}
\newcommand\ampnt{\amp^{\asa{1}{1/2}}}
\newcommand\ampntb{\amp^{\alpha^{3/2}}}
\newcommand\ampnpot{\amp^{\asa{3/2}{1/2}}}
\newcommand\ampnpotb{\amp^{\asa{1/2}{3/2}}}
\newcommand\ampnl{\amp_{\proc,~\text {Loop}}^{\asa{2}{1/2}}}
\newcommand\ampnlb{\amp_{\proc,~\text {Loop}}^{\asa{1}{3/2}}}
\newcommand\ampsq{{\cal M}}
\newcommand\ampsqmt{\ampsq^{(m,0)}}
\newcommand\ampsqnt{\ampsq^{(n,0)}}
\newcommand\ampsqnpot{\ampsq^{(n+1,0)}}
\newcommand\ampsqnl{\ampsq^{(n,1)}}
\newcommand\vampsqnl{{\cal V}^{(n,1)}}
\newcommand\hvampsqnl{\hat{\cal V}^{(n,1)}}
\newcommand\vampsqnlF{{\cal V}^{(n,1)}_{\sss FIN}}
\newcommand\hvampsqnlF{\hat{\cal V}^{(n,1)}_{\sss FIN}}
\newcommand\tampsq{\widetilde{\cal M}}
\newcommand\tampsqnt{\tampsq^{(n,0)}}
\newcommand\tampsqnpot{\tampsq^{(n+1,0)}}
\newcommand\rone{r_{[1]}}
\newcommand\rtwo{r_{[2]}}
\newcommand\Ione{\ident_1}
\newcommand\Itwo{\ident_2}
\newcommand\xii{\xi_i}
\newcommand\yij{y_{ij}}
\newcommand\phii{\varphi_i}
\newcommand\yi{y_i}
\newcommand\xic{\left(\frac{1}{\xii}\right)_c}
\newcommand\lxic{\left(\frac{\log\xii}{\xii}\right)_c}
\newcommand\omyijd{\left(\frac{1}{1-\yij}\right)_\delta}
\newcommand\omyid{\left(\frac{1}{1-\yi}\right)_\delta}
\newcommand\opyid{\left(\frac{1}{1+\yi}\right)_\delta}
\newcommand\Dfun{{\cal D}}
\newcommand\Sfun{{\cal S}}
\newcommand\Sfunij{\Sfun_{ij}}
\newcommand\asfun{a_{\Sfun}}
\newcommand\bsfun{b_{\Sfun}}
\newcommand\stepf{\Theta}
\newcommand\phsp{d\phi}
\newcommand\phspn{\phsp_{n}}
\newcommand\phspnpo{\phsp_{n+1}}
\newcommand\tphsp{d\widetilde{\phi}}
\newcommand\tphspn{\tphsp_{n}}
\newcommand\tphspnij{\tphsp_{n}^{ij}}
\newcommand\asotwopi{\frac{\as}{2\pi}}
\newcommand\gs{g_{\sss S}}
\newcommand\aW{\alpha_{\sss W}}
\newcommand\gW{g_{\sss W}}
\newcommand\aem{\alpha}
\newcommand\xicut{\xi_{cut}}
\newcommand\ximax{\xi_{\rm max}}
\newcommand\deltaO{\delta_{\sss O}}
\newcommand\deltaI{\delta_{\sss I}}
\newcommand\NC{N_{\sss c}}
\newcommand\CA{c_{\sss A}}
\newcommand\CF{c_{\sss F}}
\newcommand\TF{T_{\sss F}}
\newcommand\DA{D_{\sss A}}
\newcommand\eikint{{\cal E}}
\newcommand\eikintD{\hat{\cal E}}
\newcommand\APdamp{\overline{P}}
\newcommand\Qdamp{\overline{Q}}
\newcommand\Qop{\vec{Q}}
\newcommand\JetsB{J^{\nlightB}}
\newcommand\JetsR{J^{\nlightB+1}}
\newcommand\kin{\left\{k_k\right\}}
\newcommand\velkl{v_{kl}}
\newcommand\alkl{\alpha_{kl}}
\newcommand\avg{{\cal N}}
\newcommand\symm{\varsigma}
\newcommand\symmnij{\symm_{ij}^{(n)}}
\newcommand\symmnpoij{\symm_{ij}^{(n+1)}}
\newcommand\veck{\vec{k}}
\newcommand\kbar{\bar{k}}
\newcommand\kkdotkl{k_k\mydot k_l}
\newcommand\polv{\varepsilon}
\newcommand\polP{{\cal T}}
\newcommand\polQ{{\cal W}}
\newcommand\muF{\mu_{\sss F}}
\newcommand\muR{\mu_{\sss R}}
\newcommand\clH{{\mathbb H}}
\newcommand\clS{{\mathbb S}}
\newcommand\bt{\bar{t}}
\newcommand\bq{\bar{q}}
\newcommand\bqp{\bar{q}^\prime}
\newcommand\aNLO{{\sc\small MadGraph5\_aMC@NLO}}
\newcommand\UFO{{\sc\small UFO}}
\newcommand\MLf{{\sc\small MadLoop5}}
\newcommand\ML{{\sc\small MadLoop}}
\newcommand\CutTools{{\sc\small CutTools}}
\newcommand\OL{{\sc\small OpenLoops}}
\newcommand\MadFKS{{\sc\small MadFKS}}
\newcommand{\Ht}{H_{\sss T}}
\newcommand{\GeV}{\si{\GeV}}
\newcommand{\TeV}{\si{\TeV}}
%\newcommand{\invab}{\si{\per \atto\barn}}
 \newcommand{\ght}{g_{h t \bar t}}
\newcommand{\ghht}{g_{hh t \bar t}}
\newcommand{\cuh}{ C_{t\phi} }
\newcommand{\cbh}{ C_{b\phi}}
\newcommand{\cqu}{C_{Qt}}
\newcommand{\cquqd}{C_{QtQb}}
\newcommand{\eps}{\epsilon}
\newcommand{\as}{\frac{\alpha_s^0}{4 \pi} }
\newcommand{\Zas}{ Z_{\alpha_s} }
\newcommand{\asr}{\frac{\alpha_s}{4 \pi} }
\newcommand{\Red}[1]{{\color{nicered}{#1}}}
%%%%%%%%%%%%% functions for eff  %%%%%%%%%%%
% short-hands
\newcommand*{\mg}{\texttt{MG5\_aMC@NLO}}
\newcommand*{\fb}{\text{fb}}
\newcommand*{\iab}{\ensuremath{\text{ab}^{-1}}}
\DeclareMathOperator*{\cov}{cov}
\newcommand{\inab}{\,{\rm ab}^{-1}}
\newcommand{\infb}{\,{\rm fb}^{-1}}
\newcommand{\bbh}{b\bar bh}
\newcommand{\ggh}{gg\to h}
\newcommand{\bbaa}{b\bar b\gamma\gamma}
\newcommand{\tkab}{\tilde{\kappa}_b}
\newcommand{\kab}{\kappa_b}
\DeclareMathOperator{\BR}{BR}
\newcommand{\La}{\mathcal{L} }
\newcommand{\hhbox}{hh^{gg\rm F}_{\rm box}}
\newcommand{\hhtri}{hh^{gg\rm F}_{\rm tri}}
\newcommand{\hhint}{hh^{gg\rm F}_{\rm int}}
\newcommand{\qqA}{q\bar q \rm A}
\newcommand{\uuA}{u\bar u \rm A}
\newcommand{\ddA}{d\bar d \rm A}
\newcommand{\QQh}{Q\bar Q h}
\newcommand{\HEPfit}{\texttt{HEPfit}}

%%Maths stuff 
\newcommand{\parenths}[1]{\left({#1}\right)\xspace}
\newcommand{\braces}[1]{\left\{{#1}\right\}\xspace}
\newcommand{\sqbracs}[1]{\left[{#1}\right]\xspace}
\newcommand{\colvector}[1]{\begin{pmatrix}#1\end{pmatrix}\xspace}
\newcommand{\expec}[1]{\langle #1\rangle}
\newcommand{\Zp}{Z^\prime}
\newcommand{\s}{\hat{s}}
% number of expected events theoretical 289
%%%
\def\beq{\begin{equation}}
	\def\bea{\begin{eqnarray}}
		\def\eeq{\end{equation}}
	\def\eea{\end{eqnarray}}
\def\beqnl{\begin{align}}
	\def\endal{\end{align}}
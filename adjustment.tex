\KOMAoptions{numbers=noenddot}
\usepackage{amsmath,amssymb,amsfonts,amsthm,epigraph,scrpage2}
\usepackage[ngerman,english]{babel}
\usepackage{centernot} % f�r Durchstreichung
\usepackage{multirow}
\usepackage{pdfpages}


\usepackage{currvita}
\renewcommand*{\cvheadingfont}{\large\bfseries} % CV Titel
\renewcommand*{\cvlistheadingfont}{\bfseries\sffamily} % sub-�berschriften
\renewcommand*{\cvlabelfont}{\sffamily} % items

\usepackage{placeins}
\usepackage{booktabs}
\usepackage{ragged2e}
\usepackage{etoolbox}
\usepackage{multirow}
\usepackage{amssymb}
\usepackage{bbold}
\usepackage{lieart}
\usepackage{booktabs,colortbl}
\usepackage{slashbox}
\usepackage[normalem]{ulem}
%\usepackage{pict2e}
\usepackage{url}
\usepackage{multirow}
\usepackage{siunitx}
\usepackage{xcolor}


\definecolor{Cayenne}{rgb}{0.502,0.0,0.0}
\definecolor{Steel}{rgb}{0.4,0.4,0.4}
\definecolor{Tri_blue}{rgb}{0.039,0.5098,0.8}
\definecolor{Tri_yellow}{rgb}{0.5529,0.5451,0.0549}
\newcommand{\la}[1]{\textcolor{Cayenne}{#1}}

%\setcounter{secnumdepth}{3} % sub subsections numbering
%\setcounter{tocdepth}{3} % subsubsections inTOC

\usepackage[format=plain,singlelinecheck=false, font={sf,small},labelfont={bf,color=Steel}]{caption}
\DeclareCaptionLabelSeparator{cayenne_period}{\textcolor{Cayenne}{.} }
\captionsetup{labelsep=cayenne_period}

% Colors
\addtokomafont{chapter}{\color{Steel}}
\addtokomafont{section}{\color{Steel}}
\addtokomafont{subsection}{\color{Steel}}
\addtokomafont{subsubsection}{\color{Steel}}
\addtokomafont{paragraph}{\color{Steel}}
\addtokomafont{disposition}{\color{Steel}}
\addtokomafont{pagehead}{\color{Steel}}
\renewcommand{\pnumfont}{\color{Steel}} 
\addtokomafont{headsepline}{\color{Steel}} 
\pagestyle{scrheadings}

% Textcolor for chapters in TOC
\makeatletter
\let\stdl@chapter\l@chapter
\renewcommand*{\l@chapter}[2]{%
  \stdl@chapter{\textcolor{black}{#1}}{\textcolor{black}{#2}}}
\makeatother

%  labels in description environments
\renewcommand{\descriptionlabel}{\hspace\labelsep{}\sffamily\small\bfseries{}\color{Steel}{}}

%\makeatletter % dot after sections and all below
%\let\std@sect\@sect
%\def\@sect#1#2#3#4#5#6[#7]#8{\std@sect{#1}{#2}{#3}{#4}{#5}{#6}[#7.]{#8\color{Cayenne}{.}}}
%\makeatother
\usepackage{acronym}

\usepackage[leftcaption]{sidecap} % inner, outer,left,right
\sidecaptionvpos{figure}{t}

% Papiergr��e
%\setlength{\paperwidth}{21cm}
%\setlength{\paperheight}{25cm}
%\recalctypearea
%\usepackage[pass]{geometry}
%\usepackage[cross,a4,center]{crop}

%% Flattersatz
%\usepackage[document]{ragged2e} % Flattersatz
%\setlength{\RaggedRightParindent}{1em} % evtl. parskip


%% Sans Serif
%\usepackage{cmbright}
%\renewcommand{\familydefault}{\sfdefault}
%% Palatino
%\usepackage[sc]{mathpazo}
%\linespread{1.05}         % Palatino needs more leading (space between lines)
%\setkomafont{sectioning}{\normalcolor\bfseries} % Kapitel�berschriften

%%% Kapitel�berschriften: Mit gro�en Zahlen
%\usepackage{titlesec}
%\titleformat{\chapter}[display]
%{\bfseries\Large}
%{ %\Huge\textsc{\chaptertitlename} % f�r das Wort 'Kapitel'
%\hfill\fontsize{120}{70}\selectfont\color{lightgray}\textbf{\thechapter}}
%{-2ex}
%%{\filleft\fontsize{50}{70}\selectfont\scshape} % Kapit�lchen oder...
%{\filleft\fontsize{50}{70}\selectfont\textbf} % ...oder keine Kapit�lchen
%[\vspace{0ex}]
%
%%%% Part�berschriften
%\titleformat{\part}[display]
%{\bfseries\Large}
%{ %\Huge\textsc{\chaptertitlename} % f�r das Wort 'Kapitel'
%\hfill\fontsize{120}{70}\selectfont\color{lightgray}\textbf{\thepart}}
%{-2ex}
%{\filleft\fontsize{50}{70}\selectfont\scshape} % Kapit�lchen oder...
%%{\filleft\fontsize{50}{70}\selectfont\textbf} % ...oder keine Kapit�lchen
%[\vspace{0ex}]


\newcommand{\ER}{Erd\H{o}s-R\'enyi }
\newcommand{\BA}{Barab\'asi-Albert }
\newcommand{\mean}[1]{\left< #1 \right>}
\newcommand{\abs}[1]{\left| #1 \right|}
\newcommand{\norm}[1]{\lVert#1\rVert}
\newcommand{\mat}[1]{\mathbf{#1}}
\newcommand{\tgraph}{\mathcal{G}}

\theoremstyle{definition} % non-italic
\newtheorem{annahme}{Annahme} % braucht amsthm
\newtheorem{definition}{Definition}
\newtheorem{theorem}{Theorem}
\newtheorem{satz}{Satz}
\newtheorem{frage}{Frage}
%\input{watermarks/watermark.tex}
\DeclareMathOperator{\nnz}{nnz}

% + Graphicspath nach begin document

% aus Doi hyperref machen
%\newcommand*{\doi}[1]{\href{http://dx.doi.org/\detokenize{#1}}{doi: \detokenize{#1}}}

% Colour Scheme
\definecolor{myblue}{HTML}{003366}

% Hyperlinks Setup
\hypersetup{
    unicode=false,          
    pdftoolbar=true,        
    pdfmenubar=true,        
    pdffitwindow=false,     
    pdfstartview={FitH},    
    pdftitle={Resurrecting bbh with kinematic shapes},
    pdfauthor={Lina Alasfar, Ramona Groeber, Christophe Grojean, Ayan Paul and Zhuoni Qian},
    pdfkeywords={Future Colliders} {dihiggs} {Machine Learning} {Shapley Values},
    pdfnewwindow=true,
    colorlinks=true,
    linkcolor=myblue,
    citecolor=myblue,
    filecolor=myblue,
    urlcolor=myblue,
    linktocpage=true
}
\newcommand{\SM}{\mathrm{SM}}
\newcommand{\hc}{\mathrm{h.c.}}
%%%%%%%%%%%%%%%%%%%%%%%%%%%%%%%%%%%%%%%%
% autoref configuration
\renewcommand{\sectionautorefname}{section}
\renewcommand{\subsectionautorefname}{section}
\renewcommand{\appendixautorefname}{appendix}
\renewcommand{\tableautorefname}{table}
\renewcommand{\figureautorefname}{figure}
\def\equationautorefname~#1\null{Eq.\,(#1)\null}
\newcommand{\appendixref}[1]{\hyperref[#1]{appendix~\ref{#1}}}
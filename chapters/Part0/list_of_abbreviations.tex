%\chapter*{}
\section*{List of abbreviations}
%\addcontentsline{toc}{chapter}{List of abbreviations}
%\thispagestyle{plain}
\begin{acronym}[nnzzzzzzz] %5 l�ngste Abk�rzung ineckigen Klammern zur Ausrichtung
%\setlength{\itemsep}{-\parsep}
\acro{}[\color{Steel}{Static networks}\color{Cayenne}{.}]{}
\acro{G}[$G$]{Network/Graph. A tuple $G=(V,E)$ of a set of nodes $V$ and a set of edges~$E$.}
\acro{N}[$N$]{Number of nodes of a network.}
\acro{m}[$m$]{Number of edges of a network.}
\acro{D}[$D$]{Network diameter.}
\acro{A}[$\mathbf{A}$]{Adjacency matrix.}
\acro{P}[$\mathbf{P}_{N-1}$]{Accessibility matrix.}
\acro{Gstar}[$G^*_n$]{Accessibility graph up to path length $n$. The transitive closure is given by $G^* _{N-1}\equiv G^*$.}
\acro{exists_path}[$u\rightarrow v$]{A path of arbitrary length exists between $u$ and $v$.}
\acro{k}[$k, k^+,k^-$]{Degree of a node, Out-degree, In-degree.}
\acro{gcc}[$G(S)CC$]{Giant (strongly) connected component.}
\acro{gwcc}[$GWCC$]{Giant weakly connected component.}
\acro{lcc}[$L(S)CC$]{Largest (strongly) connected component. Often used synonymous for G(S)CC.}
\acro{Q}[$Q$]{Modularity.}

\acro{}[]{}
\acro{}[\color{Steel}{Epidemic models}\color{Cayenne}{.}]{}
\acro{alpha}[$\alpha $]{Infection rate.}
\acro{gamma}[$\gamma $]{Recovery rate.}
\acro{rinfty}[$R_\infty $]{Outbreak size in SIR model.}

\acro{}[]{}
\acro{}[\color{Steel}{Temporal networks}\color{Cayenne}{.}]{}
\acro{tmpgraph}[$\mathcal{G}$]{Temporal network given by triple $\mathcal{G}=(V,\mathcal{E},T)$.}
\acro{adjmatrixseq}[$\mathcal{A}$]{Sequence of adjacency matrices as a graph centric temporal network representation.}
\acro{pn}[$\mathcal{P}_n $]{Accessibility matrix of a temporal network over $n$ time steps.}
\acro{Gstartemp}[$\mathcal{G}^*_n$]{Accessibility graph up to path \emph{duration} $n$. The real fully unfolded accessibility graph is in general $\mathcal{G}^*\equiv \mathcal{G}^*_\infty $.}
\acro{exists_temp_path}[$u\rightsquigarrow v$]{A time respecting (causal) path exists between $u$ and $v$.}
\acro{t_horizon}[$\mathcal{H}_v$]{Horizon of node $v$.}
\acro{nnz}[$\mathrm{nnz}(\mat{X})$]{Number of non zeros of a matrix $\mat{X}$.}
\acro{rho}[$\rho (\mathbf{X})$]{Density of a matrix $\mat{X}$, i.e. the number of occupied non zeros normalized by the number of all possible entries.}
\acro{ranking}[$R(Y)$]{Node ranking according to some measure $Y$.}
\acro{infperiod}[$d$]{Infectious period.}
\acro{}[$r(v,d,t_0)$]{Range of a node for memory/infectious period $d$ and starting time $t_0$. Equivalent to outbreak size for simple compartment models.}
\acro{}[$\mathcal{S}$]{Set of outbreak scenarios containing elements of the form $(v,d,t_0,r(v,d,t_0))$.}

\acro{}[]{}
\acro{}[\color{Steel}{Randomization models}\color{Cayenne}{.}]{}
\acro{}[$RE$]{Randomized edges model. Each $\mathbf{A}$ in $\mathcal{A}$ is randomized so that the degree of each nodes is preserved.}
\acro{}[$TR$]{Time reversal. All edges and the order of matrices in $\mathcal{A}$ are reversed.}
\acro{}[$GST$]{Globally shuffled times. The sequence $\mathcal{A}$ is rearranged in random order.}
\acro{}[$LST$]{Locally shuffled times. All edge occurrence times are placed randomly and the number of occurrences is preserved.}
\acro{}[$RT$]{Random times. Every snapshot of $\mathcal{G}$ is taken as a random subset of the aggregated network.}



\end{acronym}

\selectlanguage{english}
\begin{abstract}
Since the discovery of the Higgs boson in 2012 at the Large Hadron Collider~(LHC), most of its properties have been already measured with increasing accuracy. However, few of the Higgs's main properties are yet to be probed.  The future runs of the LHC hold a lot of potential for further understanding of the Higgs boson's properties.
\par Many of Higgs production processes are plagued by large theoretical uncertainties. In order to reduce them, these processes need to be computed at higher precision. One of such processes is the production of the Higgs with he $Z$ boson.  This is a key process in measuring both the Higgs mass and its couplings with more precision.  My collaborators and I have preformed such calculation using a novel method that can be used alongside other calculations via Pad\'e approximants and then incorporated into Monte Carlo event simulations~\cite{Alasfar:2021ppe}.
\par In an another project, we have studied the correlations between the Higgs self-interaction and other set of interactions involving four heavy quarks.  Both interaction classes are equally weakly constrained from current LHC data. Using Markov-chain Monte Carlo~(MCMC) Bayesian analysis, we have seen that there is a strong correlations amongst these observables. Revealing many challenges to probing Higgs self-coupling using current Higgs measurements~\cite{Alasfar:2021pmn}. 
\par The only direct way to study the Higgs self-coupling is to search for Higgs bosons produced in pairs.  A process that is sought after in the High-Luminosity~(HL)-LHC. We have used interpretable machine learning  to improve upon the expected sensitivity of the HL-LHC and future colliders to this process. We were able to constrain both the Higgs's self interaction and its interaction with light quarks, and show that they are uncorrelated from this process~\cite{Alasfar:2019pmn,Alasfar:2021xdd}. 
\par Since 2015, experimental data was hinting towards an anomaly involving the decay of composite particles known as $B$-mesons that could be a result of new physics beyond the Standard Model.  We have used~(MCMC) Bayesian analysis to show that the parameters characterising new physics  in these decays are strongly related to other set of parameters in the Standard Model related to the interaction between the $Z$ and Higgs bosons, and muons~\cite{Alasfar:2020mne}.

\paragraph{Keywords\color{Cayenne}{:}} Higgs Physics, Standard Model Effective Field Theory, Flavour observables, Statistical data analysis
\end{abstract}

\cleardoublepage


\selectlanguage{ngerman}
\begin{abstract}

%\vspace{-0.2cm}
%\vfill
%\noindent \textbf{Schlagw�rter}\color{Cayenne}{:} Netzwerk, Epidemiologie, zeitabh�ngiges Netzwerk
\paragraph{Schlagw\"orter\color{Cayenne}{:}} Higgs Physik, Standardmodell Effektive Feldtheorie,  Flavour Anomalies, Statistische Datenanalyse
\end{abstract}

% Back to main language
\selectlanguage{english}
\cleardoublepage
%!TEX encoding = UTF-8 Unicode
% !TeX spellcheck = en_GB

\selectlanguage{english}
\begin{abstract}
This thesis investigates some future aspects of Higgs measurements a decade after its discovery, focusing on the potential for future runs of the Large Hadron Collider~(LHC).  In particular, it aims to probe challenging couplings of the Higgs like its self-coupling and interaction with light quarks.\\ The first part provides an overview of Higgs theory and measurements, with some meta-analysis of most recent results focusing on the Standard Model Effective Field theory~(SMEFT). The second part is about single-Higgs production, starting with a two-loop calculation of the gluon fusion component of $Zh$ to reduce its theoretical uncertainties. Then, the potential for constraining the Higgs trilinear self-coupling from single Higgs rates is revisited; by including equally weakly-constrained four-heavy-quark operators entering at the next-to-leading order in single Higgs rates.  These operators highly correlate with the trilinear self-coupling, thus affecting the fits made on this coupling from single Higgs data.  \\ The third part focuses on the Higgs pair production, an essential process for measuring Higgs-self coupling, employing multivariate analysis to study its potential for probing light Yukawa couplings. Thereby exploring the sensitivity of Higgs pair production for the light-quark Yukawa interactions.\\ Finally, the fourth part showcases some models aiming to explain the recent flavour anomalies in the light of a global SMEFT Bayesian analysis combining flavour and electroweak precision measurements. 

\paragraph{Keywords\color{Cayenne}{:}} Higgs Physics, Standard Model Effective Field Theory, Flavour observables, Statistical data analysis.
\end{abstract}

%\cleardoublepage


\selectlanguage{ngerman}
\begin{abstract}
	fuer spaeter
%Diese Dissertation untersucht einige Aspekte f\"ur die Zukunft der Higgs-Messungen nach einem Jahrzehnt seiner Entdeckung, im Rahmen der Zukunft des LHC's\\ Der erste Teil bietet einen \"Uberblick über die Higgsphysik und -Messungen, mit einigen Metaanalysen zu den aktuellen Ergebnissen und einem Schwerpunkt auf der Standardmodell-Effektivfeld-Theorie (SMEFT). Der zweite Teil befasst sich mit der Einzel-Higgs-Produktion.  Zus\"atzlich werden Ergebnisse f\"ur $Zh$-Produktion via Gluon-Fusion in n\"achstf\"uhrender Ordnung in der starken Kopplungskonstante im Niederenergielimes präsentiert. Dann eine SMEFT-Analyse der Interaktion zwischen Higgsselbstkopplung und vier Heavy-Quark-Operatoren, die von Korrekturen h\"oherer Ordnung stammen. \\ Der dritte Teil konzentriert sich auf die Higgspaarproduktion, ein wesentlicher Prozess zur Messung der Higgsselbstkopplung. Unter Verwendung multivariater Analyse zur Untersuchung seines Potenzials zur Untersuchung leichter Yukawa-Kopplungen. \\ Schlie\ss lich werden einige Modelle vorgeschlagen, um die letzten Flavour-Anomalies im inspiriert von einer globalen SMEFT-Bayesschen Analyse zu erkl\"aren.
%\vspace{-0.2cm}
%\vfill
\paragraph{Schlagw\"orter\color{Cayenne}{:}} Higgs Physik, Standardmodell-Effektivfeld-Theorie,  Flavour Anomalies, Statistische Datenanalyse
\end{abstract}

% Back to main language
\selectlanguage{english}
\cleardoublepage
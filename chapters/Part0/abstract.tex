%!TEX encoding = UTF-8 Unicode
% !TeX spellcheck = en_GB

\selectlanguage{english}
\begin{abstract}
This thesis investigates some aspects for the future of Higgs measurements after a decade of its discovery, focusing on the potential for the future runs of the Large Hadron Collider. \\ The first part provides  an overview of the Higgs theory and measurements, with some meta-analysis on the most recent results and focus on the Standard Model Effective Field theory~(SMEFT). The second part is concerned with single-Higgs production, and two-loop calculation of $Zh$ production via gluon fusion. Then a SMEFT analysis of the interplay between Higgs self-coupling and four heavy quark operators stemming from Higher order effects. \\ The third part focuses on the Higgs pair production, an essential process for measuring Higgs-self coupling. Employing multivariate analysis to study its potential for probing light Yukawa couplings.\\ Lastly, some models aims to explain the recent flavour anomalies are proposed, in the light of a global SMEFT Bayesian analysis. 

\paragraph{Keywords\color{Cayenne}{:}} Higgs Physics, Standard Model Effective Field Theory, Flavour observables, Statistical data analysis.
\end{abstract}

%\cleardoublepage


\selectlanguage{ngerman}
\begin{abstract}
Diese Dissertation untersucht einige Aspekte f\"ur die Zukunft der Higgs-Messungen nach einem Jahrzehnt seiner Entdeckung, im Rahmen der Zukunft des LHC's\\ Der erste Teil bietet einen \"Uberblick über die Higgsphysik und -Messungen, mit einigen Metaanalysen zu den aktuellen Ergebnissen und einem Schwerpunkt auf der Standardmodell-Effektivfeld-Theorie (SMEFT). Der zweite Teil befasst sich mit der Einzel-Higgs-Produktion.  Zus\"atzlich werden Ergebnisse f\"ur $Zh$-Produktion via Gluon-Fusion in n\"achstf\"uhrender Ordnung in der starken Kopplungskonstante im Niederenergielimes präsentiert. Dann eine SMEFT-Analyse der Interaktion zwischen Higgsselbstkopplung und vier Heavy-Quark-Operatoren, die von Korrekturen h\"oherer Ordnung stammen. \\ Der dritte Teil konzentriert sich auf die Higgspaarproduktion, ein wesentlicher Prozess zur Messung der Higgsselbstkopplung. Unter Verwendung multivariater Analyse zur Untersuchung seines Potenzials zur Untersuchung leichter Yukawa-Kopplungen. \\ Schlie\ss lich werden einige Modelle vorgeschlagen, um die letzten Flavour-Anomalies im inspiriert von einer globalen SMEFT-Bayesschen Analyse zu erkl\"aren.
%\vspace{-0.2cm}
%\vfill
\paragraph{Schlagw\"orter\color{Cayenne}{:}} Higgs Physik, Standardmodell-Effektivfeld-Theorie,  Flavour Anomalies, Statistische Datenanalyse
\end{abstract}

% Back to main language
\selectlanguage{english}
\cleardoublepage
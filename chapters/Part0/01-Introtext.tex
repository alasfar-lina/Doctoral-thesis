%!TEX encoding = UTF-8 Unicode
% !TeX spellcheck = en_GB


\chapter{Introduction}
Since the discovery of the Higgs particle in 2012 by collaborative efforts between ATLAS~\cite{ATLAS:2012yve} and CMS~\cite{CMS:2012qbp} experiments at the { L}arge { H}adron {C}ollider~(LHC), and the Standard Model of particle physics~(SM)~\cite{salam1,salam2,PhysRevLett.19.1264}, has been completed~\cite{PhysRevLett.13.321,PhysRevLett.13.508,HIGGS1964132,PhysRevLett.13.585,Guralnik:2009jd}, albeit leaving us with more questions than what is has answered. The most prominent question was: \textit{What are the properties of this newly-discovered particle ?}\\
%
Answering this very question has become the preeminent goal of the LHC. Higgs measurements are getting progressively accurate, and our understanding of this particle is approaching few percent-level. The future runs of the LHC, will open the doors of the Higgs-precision era. However, increased luminosity, i.e. data acquisition from the LHC, without improving the theoretical prediction of Higgs processes is futile. Therefore, to insure the success of the experimental efforts in probing Higgs couplings and properties at the required fastidiousness, it is imperative to include higher-order calculations for Higgs production cross-sections. \\ 
An example of such processes, is the associated production of Higgs with $Z$ bosons, which suffers from higher theoretical uncertainties compared to its sister process, the $Wh$ production, due to the presence of the gluon fusion sub-process $ gg \to Zh$. These uncertainties can be improved by including the two-loop corrections to the gluon fusion sub-process efficiently using state-of-the-art analytic techniques~\cite{Bonciani:2018omm}. 
%
 \\ After a decade of \emph{Higgs physics}, and over ten-thousand Higgs-related publications, we still have a lot to learn about the Higgs boson.  In particular,  the structure of its potential is still unknown, as well as its couplings to the light quarks and leptons. Measurements of Higgs self-coupling, will reveal to us if there are new scalars beyond the Higgs boson which we have not yet directly observed. Furthermore, studying Higgs coupling to light fermions is essential in understanding the source of their masses, and explaining the significant hierarchy between these across the three generations of matter. \\ 
 The High Energy Physics community, anticipated that the Higgs discovery, especially given its mass is $ m_h <130$ GeV, will be followed by the detection of a \emph{Zoo} of particles stemming from Supersymmetric extension of the SM, cf.~\cite{handle:20.500.11811/5384}. Unfortunately, this was not the case, and we have not seen any new particles discovered after the Higgs boson. This motivated parametrising new physics~(NP) effects in a model-independent manner,  within what-so-called the Standard Model Effective Field Theory~(SMEFT) framework~\cite{Grzadkowski:2010es,Alonso:2013hga}. In this formalism, all NP interactions are summarised in 56 terms, which make very little assumptions about  the nature of NP, guaranteeing a model-independent approach to collider searches. \\ 
 %
 The use of SMEFT in higher-order calculations of Higgs rates has revealed insights on the Higgs potential by the appearance of Higgs trilinear self-coupling within electroweak loop corrections of single-Higgs processes ~\cite{DiVita:2017vrr} . With the help of precision measurements of single-Higgs rates at the LHC, the trilinear coupling could be constrained~\cite{DiMicco:2019ngk} from single-Higgs data. Nevertheless, more SMEFT observables can also enter in single-Higgs loops that alters the constraining power of these measurements. In fact, the interconnectivity between the Higgs and top-quark sectors are further emphasized within the SMEFT framework, as recent global fits have established strong correlations between observables from both sectors as well as the electroweak precision observables~(EWPO)~\cite{Ellis:2020unq}. Strong correlations between the top and EWPO are also seen at loop-level~\cite{Dawson:2020oco}, thus one expects to see similar correlations emerging from loop effects of top operators on Higgs processes. \\
%%
 The observation of Higgs pairs is slated for the High-Luminosity (HL) LHC operating phase. This rare process will be --if observed-- the pi\`ece de r\'esistance of the LHC Higgs physics programme~\cite{Bernius:2666331}, directly measuring the Higgs self-interaction. Also, untangling Higgs potential measurements  from the top-sector interactions. Furthermore,  this process could be of great utility in probing Higgs coupling to light quarks, as well. The full potential of Higgs pair production can be exploited when it is treated an a multivariate problem by implementing interpretable machine learning analysis technique~\cite{Grojean:2020ech}. In this manner, it is possible to have a simultaneous measurement of the two most elusive Higgs interactions, the trilinear self-coupling and light Yukawas.\\
 %%
 Recent measurements, by  Belle and Babar, in addition to the LHCb experiment at CERN, of $B$-mesons semi-leptonic decays showed some tension with the SM predictions of lepton flavour universality of electroweak couplings~\cite{Aaij:2014ora,Aaij:2017vbb,Aaij:2019wad,Abdesselam:2019wac,LHCb:2021trn}, with up to $\sim 3\sigma$ deviation from the SM~\cite{Chatrchyan:2013bka,Aaij:2017vad,Aaboud:2018mst,Aaij:2020nol}. These  anomalies require models with some flavour violation, that makes model-building for explaining these anomalies az tree-level an Augean task~\cite{DiLuzio:2017vat,Calibbi:2017qbu,Bordone:2017bld,Barbieri:2017tuq,Assad:2017iib,Heeck:2018ntp,Fornal:2018dqn,Crivellin:2018yvo,Crivellin:2019dwb,Bordone:2019uzc}. Additionally, to complicate things further, these anomalies are in dissension with EWPO. Hence, promoting for a more careful treatment of these anomalies; introducing them at loop-level in SMEFT and preforming a global fit combining both flavour and EWPO data. Thus allowing for a SMEFT guided UV-model building for these anomalies, with extended Higgs and top sectors. \\
 
\paragraph{This thesis is structured as follows\color{Cayenne}{:}}
We start by an introduction to the theory of the Higgs field and its role in the SM in~\autoref{chap:HiggsSM}, followed by a theoretical constraints on the Higgs. In~\autoref{chap:HiggsConstr}, a review of the state-of-the-art Higgs measurements and the constraints on Higgs couplings derived from the latest LHC data. After that, I present the basics of Effective Fields Theories relevant to Higgs physics at the LHC in~\autoref{chap:HiggsEFT}. \\ The second part of the thesis focuses on the production of --single-- Higgs at the LHC, starting with an overview in~\autoref{chap:overviewSingleHiggs}, followed by a discussion on the use of $\pt$-expansion technique for obtaining an analytic expression for the virtual correction of the gluon fusion $Zh$ production in~\autoref{chap:hz}. Next,\autoref{chap:4topSingleHiggs} showcases the potential for single-Higgs processes to probing four-fermion operators from the top sector of SMEFT, along with the trilinear self-coupling by preforming Higher-order computations of these processes in SMEFT, followed by a Bayesian fit. \\ The third part of the thesis focuses on the production of Higgs in pairs, namely at the HL-LHC in~\autoref{chap:overviewDiHiggs}. Then, I show the potential for employing Higgs pair production to probe light quark coupling to the Higgs. In addition the methods of multivariate analysis to maximise the efficiency of extracting the Higgs pair signal using interpretable machine learning.  The last part of the thesis, has only~\autoref{chap:flav} and it is about potential UV models for the $B$ anomalies, inspired by a global SMEFT fit and minimal flavour violation~(MFV). 










%!TEX encoding = UTF-8 Unicode
% !TeX spellcheck = en_GB


\chapter{Introduction}
The discovery of the Higgs boson in 2012 by the ATLAS~\cite{ATLAS:2012yve}  and CMS~\cite{CMS:2012qbp} experiments at the {L}arge {H}adron {C}ollider~(LHC) marks the completion of the Standard Model of particle physics~(SM)~\cite{salam1,salam2,PhysRevLett.19.1264}; as it was a direct prediction of the spontaneous symmetry breaking mechanism observed in the SM~\cite{PhysRevLett.13.321,PhysRevLett.13.508,HIGGS1964132,PhysRevLett.13.585,Guralnik:2009jd}. However, this discovery has brought more questions than answers, and even after a decade of its discovery, there is a lot to know about this particle and its potential connections with physics beyond the SM.
%
Understanding the properties and couplings of the Higgs boson has become the preeminent goal of the LHC. Higgs measurements are getting progressively accurate, and our understanding of this particle is approaching a few per cent-level. The future runs of the LHC will open the doors to the Higgs-precision era. However, increased luminosity, i.e. data acquisition from the LHC, without improving the theoretical prediction of Higgs processes is futile. Therefore, to ensure the success of the experimental efforts in probing Higgs couplings and properties at the required precision, it is imperative to include higher-order calculations for Higgs production cross-sections. \\ 
An example of such processes is the associated production of the Higgs boson with a $Z$ boson, which suffers from higher theoretical uncertainties than its sister process, the $Wh$ production, because it contains a gluon fusion sub-process $ gg \to Zh$. Furthermore, the gluon fusion channel generally tends to have large higher-order corrections compared to the quark-initiated one;  thereby, prompting the need to compute its higher order corrections, in order to improve the theoretical prediction of  $Zh$ production. Such computation can be carried out efficiently using a state-of-the-art analytic technique based on the expansion in small transverse momentum proposed in ref.~\cite{Bonciani:2018omm}. 
%
\\ After a decade of \emph{Higgs physics}, and over ten-thousand Higgs-related publications, we still have a lot to learn about the Higgs boson.  In particular, its potential structure is yet to be probed experimentally, and so are its couplings to the light quarks and leptons. Measurements of Higgs self-coupling will reveal if there are, for instance, new scalars beyond the Higgs boson that we have not yet directly observed. Furthermore, studying Higgs coupling to light fermions is essential in understanding the source of their masses' origin and explaining the significant hierarchy between these across the three generations of matter. \\ 
%The High Energy Physics community anticipated that the Higgs discovery, especially given its mass is $ m_h <130$ GeV, will be followed by the detection of a \emph{Zoo} of particles stemming from the Supersymmetric extension of the SM. Unfortunately, this was not the case, and we have not seen any new particles discovered after the Higgs boson. 
The conclusion of the SM-related discoveries did not leave any specific hints to the nature and scale of new physics~(NP).  Moreover, many experimental searches have excluded NP at scale close to the electroweak symmetry breaking, for most recent searches cf.~\cite{ATLAS-CONF-2022-006,ATLAS-CONF-2022-011,ATLAS-CONF-2022-012,ATLAS-CONF-2022-010,ATLAS-CONF-2022-009,CMS-PAS-EXO-20-011,CMS-PAS-EXO-21-010,CMS-PAS-EXO-21-003,CMS-PAS-EXO-20-006,CMS-PAS-EXO-21-006,CMS:2022nty,CMS:2022yjm}. Although NP is needed to explain the shortcomings of the SM as for instance: neutrino masses, or give a candidate for dark matter and so on. Experimental searches have excluded for most scenarios that NP is at a scale close to the electroweak symmetry breaking.
This motivates parametrising NP effects in a model-independent manner, in terms of higher-dimensional operators suppressed by some high scale~$\Lambda$ . This formalism is known as the Standard Model Effective Field Theory~(SMEFT) framework~\cite{Giudice:2007fh,Grzadkowski:2010es,Contino:2013kra, Elias-Miro:2013eta,Gupta:2014rxa}. In SMEFT, all leading NP effects in Higgs physics are summarised in a numerable set of mass dimension six operators, that makes minimal assumptions about the nature of NP, guaranteeing a model-independent approach to collider searches.\\ 
%
The use of SMEFT in higher-order calculations of Higgs rates has revealed insights into the Higgs potential by the appearance of the Higgs trilinear self-coupling within electroweak loop corrections of single-Higgs processes. This allows to put constraint on this coupling from measurements of single-Higgs rates at the LHC can be used to constrains this coupling~\cite{McCullough:2013rea, Gorbahn:2016uoy, Degrassi:2016wml, Bizon:2016wgr, Maltoni:2017ims, Degrassi:2019yix, Degrassi:2021uik, Haisch:2021hvy}. Nevertheless, more SMEFT operators can also enter in single-Higgs loops that alter the constraining power of these measurements. The interconnectivity between the Higgs and top-quark sectors is  emphasised within the SMEFT framework, as recent global fits have established strong correlations between observables from both sectors as well as the electroweak precision observables~(EWPO)~\cite{Ellis:2020unq}. Strong correlations between the top sector and EWPO are also seen at loop-level~\cite{Dawson:2020oco,Dawson:2022bxd} thus; one expects to see similar correlations emerging from loop effects of top operators on Higgs processes. \\
%%
The observation of Higgs pairs is slated for the High-Luminosity (HL) LHC operating phase. This rare process will be --if observed-- the {\it pi\`ece de r\'esistance} of the LHC Higgs physics programme~\cite{Bernius:2666331}, directly measuring the Higgs trilinear self-interaction, also untangling Higgs potential measurements from the top-sector interactions. Furthermore,  this process could be of great utility in probing Higgs coupling to light quarks, from the enhancement of the quark-initiated Higgs pair production, cf.~\cite{Alasfar:2019pmn,Egana-Ugrinovic:2021uew} and as will be shown in this thesis.  The full potential of Higgs pair production can be exploited when it is treated as a multivariate problem by implementing an interpretable machine learning analysis technique~\cite{Grojean:2020ech}. In this manner, it is possible to have  simultaneous constraints of the two most elusive Higgs interactions, light-quark Yukawa and the trilinear couplings.\\
%%
Recent measurements, by Belle and Babar, in addition to the LHCb experiment at CERN, of $B$-mesons semi-leptonic decays showed some tension with the SM predictions of lepton flavour universality of electroweak couplings~\cite{Aaij:2014ora,Aaij:2017vbb,Aaij:2019wad,Abdesselam:2019wac,LHCb:2021trn}, with up to $\sim 3\sigma$ deviation from the SM~\cite{Chatrchyan:2013bka,Aaij:2017vad,Aaboud:2018mst,Aaij:2020nol}. These anomalies require models with some flavour violation that makes model-building for explaining these anomalies at tree-level  Augean task~\cite{DiLuzio:2017vat,Calibbi:2017qbu,Bordone:2017bld,Barbieri:2017tuq,Assad:2017iib,Heeck:2018ntp,Fornal:2018dqn,Crivellin:2018yvo,Crivellin:2019dwb,Bordone:2019uzc}. Additionally, to complicate things further, these anomalies are in tension with EWPO. Hence, this thesis promotes a more careful treatment of these anomalies, by introducing them at the loop level in SMEFT and performing a global fit combining both flavour and EWPO data. The fit result would allow for a SMEFT guided UV-model building for these anomalies, with extended Higgs and top sectors. \\

\paragraph{This thesis is structured as follows\color{Cayenne}{:}}
I start with an introduction to Higgs physics and its role in the SM in~\autoref{chap:HiggsSM}, followed by theoretical constraints on the Higgs boson. In~\autoref{chap:HiggsConstr}, I review of  state-of-the-art Higgs measurements and the constraints on Higgs couplings derived from the latest LHC data. After that, I present the basics of Effective Fields Theories relevant to Higgs physics at the LHC in~\autoref{chap:HiggsEFT}. \\ The second part of the thesis focuses on the production of --single-- Higgs at the LHC, starting with an overview in~\autoref{chap:overviewSingleHiggs}, followed by a discussion on the use of the $\pt$-expansion technique for obtaining an analytic expression for the virtual correction of the gluon fusion $Zh$ production in~\autoref{chap:hz}. Next, \autoref{chap:4topSingleHiggs} showcases the potential of single-Higgs processes to probe four-fermion operators from the top sector, by performing higher-order computations of these processes in SMEFT. The potential for constraining these operators for the considered single-Higgs production processes alongside the trilinear Higgs self-coupling is investigated by means of a Bayesian fit. \\ The third part of the thesis focuses on the production of Higgs boson in pairs at the HL-LHC~(\autoref{chap:overviewDiHiggs}). Afterwards, in~\autoref{chap:lightyuk}, I show the potential for employing Higgs pair production to probe light quark couplings to the Higgs boson. In addition, I show a multivariate analysis method, that maximises the efficiency of extracting the Higgs pair signal using interpretable machine learning.  The last part of the thesis, \autoref{chap:flav}, describes the potential UV models for the $B$ anomalies, inspired by a global SMEFT fit and minimal flavour violation~(MFV). 










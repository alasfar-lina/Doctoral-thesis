%!TEX encoding = UTF-8 Unicode
% !TeX spellcheck = en_GB
%%%%%%%%%%%%%%%%%%%%%%%%%%%%%%%%%%%%%%
\chapter{ Virtual 2-loop calculation of  $Zh$ production via gluon fusion}\label{chap:hz}
%%%%%%%%%%%%%%%%%%%%%%%%%%%%%%%%%%%%%%
%\section{Overview}
\par As we have seen in the previous sections, Higgs couplings to the weak vector bosons, i.e. $Z$ and $W$ is approaching the precision level. Moreover, the associated Higgs production with these bosons is the first channel used to observer the Higgs decaying into beauty quarks   $h \rightarrow b \bar{b}$ by both ATLAS and CMS~\cite{Aaboud:2018zhk, Sirunyan:2018kst}. Hence, the $ Vh$ Higgs production channel is one of the important channels to look for in the future runs of the LHC for better measurement of the $VVh$ coupling as well as Higgs coupling to the beauty quark. As the statistical and systematic uncertainties coming from the experimental setup of the LHC get reduced in the future runs, due to higher integrated luminosity and uprated detectors and analysis techniques. There is a  need to reduce theoretical uncertainties emerging from the perturbative calculations of  cross-sections. In order to achieve that, one should include more terms in the perturbative expansion in the couplings, particularly the string coupling $\alpha_s$. In this chapter, we are interested in the channel $pp\to Zh$, which is quark-initiated tree-level process at LO interpreted as \textbf{Drell-Yan process}~ \cite{Han:1991ia,Brein:2003wg}. This process has been computed up to next-to-next-to-leading-order (NNLO) in QCD ($\sim \alpha_s^2$), and
at next-to-leading-order (NLO) in the EW interactions ($\sim \alpha^2 $) \cite{Amoroso:2020lgh}.
%%
\par Despite arising for the first time at NNLO in perturbation theory to the partonic cross-section  , the gluon fusion channel $g g \rightarrow Zh$ has a non-negligible contribution to the hadronic cross-section of  $pp\to Zh$ process, which could reach $>16\%$ of the total cross-section contribution at $14$ TeV~\cite{Cepeda:2019klc}, see~\autoref{fig:hzratio} .The contribution becomes more significant when looking at large invariant mass bins in the differential cross-section. This is due to the significant abundance of gluons at the LHC for large $Q$ as well as the top quark initiated contribution near the $t\bar t$ threshold~\cite{Englert:2013vua}.  The gluon fusion channel has a higher scales uncertainties than the quark induced one, and due to the significant contribution of the former, and the absence of gluon fusion channel for $Wh$ channel, the $Zh$ channel has  higher theoretical uncertainties. This motivates NLO calculation of the  $g g \rightarrow Z h$ channel in order to reduce these uncertainties and facilitate the precision measurement potential of the $Zh$ channel at the future LHC runs, such as sign and magnitude
of the top Yukawa coupling,  dipole operators \cite{Englert:2016hvy}
and it can receive additional contributions from new particles \cite{Harlander:2013mla}. Therefore, better understanding of the SM prediction of the $Zh$ gluon fusion channel is crucial for both the SM precision measurements of Higgs production within the SM and for testing NP in this channel, e.g. new vector-like leptons.  
%%
\begin{figure}
	\begin{center}
		\includegraphics[width=9cm]{./figures/Rplot}
		\caption{The ratio of the $LO$ gluon fusion production cross-section $ gg \to Zh$  $\sigma_{gg}$ with respect to the $NLO$ Drell-Yan process $ q\bar{q} \to Zh$ cross section $\sigma_{q\bar{q}}$ at a $pp$ collider with centre of mass energy $\sqrt{s}$. The error band captures the total theoretical uncertainties on both cross-sections.}
		\label{fig:hzratio}
	\end{center}
\end{figure}
%%
\par  The leading order (LO) contribution to the $g g \rightarrow Z H$ amplitude, given by one-loop diagrams, was computed exactly in refs.\cite{Kniehl:1990iva, Dicus:1988yh}.However, for the NLO, the virtual corrections contain multi-scale two-loop integrals some of which are still not known analytically ( for the box diagram).  The fiirst computation of the NLO terms has been done by~\cite{Altenkamp:2012sx} using an asymptotic expansion in the limit
$\mt \rightarrow \infty$ and $m_b = 0$, and pointed to a $K$-factor of about $\sim2$.  Later, the computation has been improved via soft gluon resummation, and including NLL terms found in ref.\cite{Harlander:2014wda}, the NLL terms has been matched to the fixed NLO computation of~\cite{Altenkamp:2012sx}.  Top quark mass effects to the  $g g \rightarrow Zh$ process were first implemented using a combination of of large-$\mt$ expansion (LME) and Pad\'e approximants~\cite{Hasselhuhn:2016rqt}. A data-driven approach to extract the gluon fusion dominated non-Drell-Yan part of $Zh$ production using the known relation between  $W H$
and $ Z H$ associated production when only the Drell-Yan component of the two processes is considered has been investigated in ref.\cite{Harlander:2018yns}. The differential distributions of $g g \rightarrow Zh$  at NLO was studied in ref.\cite{Hespel:2015zea} via LO matrix element matching. 
%%
\par More recent studies of the NLO virtual corrections to this process were based on the high-energy~(HE) expansion improved by Pad\'e approximants with the LME, which extended the validity range of the HE expansion \cite{Davies:2020drs}. However, this expansion is only valid for in the invariant mass region $\sqrt{\hat{s}}  \gtrsim 750\, \si{\GeV} $ and $\sqrt{\hat{s}}  \lesssim 350\,  \si{\GeV}$,  which only covers $\sim 32\%$ of the hadronic cross section. Additionally, numerical computation of the 2 loop virtual corrections, though implemented exactly in  \cite{Chen:2020gae}, are rather slow for practical use in MC simulations.  This highlights the importance of an analytical method that can cover the remaining $68\%$ region of the cross-section and can be merged with the HE expansion via Pad\'e approximants. Fortunately, the two-loop corrections to the triangle diagrams can be computed exactly. And the loop integrals appearing in the box correction having no analytic expression can be expanded in small  $Z$ (or Higgs) 
transverse momentum, $\pt$. This method was first used for Higgs pair production in~\cite{Bonciani:2018omm}, to compute the NLO virtual corrections to the box diagrams in the forward kinematics.  In this chapter, I discuss the method and results of the two-loop calculation of the triangle and $\pt$ expansion of $Zh$ process published in~\cite{Alasfar:2021ppe}. 
%%
\par This chapter is structured as follows : In~\autoref{chap6sec:GenNot} contains the general notation for the gluon fusion $Zh$ process . Then, in.. the transverse momentum expansion method is discussed.  Calculation of the LO form-factors in the transverse momentum expansion is shown in... Outline of the two-loop calculation of the triangle topology is illustrated in... Finally, in  I discuss the transverse momentum expansion for the box topology and  .. contains a brief summary and outlook. 

%%%%%%%%%%%%%%%%%%%%%%%%%%%%%%%%%%%%%%%%%%%%%%%%%
\section{General notation \label{chap6sec:GenNot} }
\par The amplitude  $g^\mu_a(p_1)g^\nu_b(p_2)\to Z^\rho(p_3) h(p_4)$ can be written as
\begin{align}
&&\amp=i \sqrt{2}\frac{\mz \Gfer \as(\mu_R)}{\pi}\delta_{ab}\epsilon^a_\mu(p_1)
\epsilon^b_\nu(p_2)\epsilon_\rho(p_3)\hat{\amp}^{\mu\nu\rho}(p_1,p_2,p_3 ),\\
&&\hat{\amp}^{\mu\nu\rho}(p_1,p_2,p_3 )=\sum_{i=1}^{6}
\mathcal{P}_i^{\mu\nu\rho}(p_1,p_2,p_3 )
\amp_i(\hat{s},\hat{t},\hat{u},\mt,\mh,\mz),
\label{eq:amp}
\end{align}
where  $\mu_R$ is the renormalisation scale and
$\epsilon^a_\mu(p_1)\epsilon^b_\nu(p_2)\epsilon_\rho(p_3)$ are the
polarization vectors of the gluons and the $Z$ boson, respectively.  It is possible to decompose the amplitude into a maximum of $6$ Lorentz structures encapsulated by the 
tensors $\mathcal{P}_i^{\mu\nu\rho}$. Due to the presence of the~$\gamma_5$ these projectors are
proportional to the Levi-Civita total anti-symmetric tensor
$\epsilon^{\alpha\beta\gamma\delta}$. One can choose to an orthogonal basis explicitly shown in~\autoref{app:uno}, such that
\begin{equation}
	\mathcal{P}_i^{\mu\nu\rho} \mathcal{P}_j\,_{\mu\nu\rho} = 0, \,\,\, \,\, \text{for}\, i \neq j 
\end{equation}
By this choice one obtains unique form factors corresponding to each projector
\begin{equation}
\amp_i(\hat{s},\hat{t},\hat{u},\mt,\mh,\mz),
\end{equation}
 that are multivariate complex functions of the
top ($\mt$), Higgs ($\mh$) and $Z$ ($\mz$) bosons masses, and of
the partonic Mandelstam variables
\begin{equation}
\hat{s}=(p_1+p_2)^2,~~ \hat{t}=(p_1+p_3)^2,~~ \hat{u}=(p_2+p_3)^2,
\end{equation}
where $\hat{s}+\hat{t}+\hat{u}=\mz^2+\mh^2$ and we took all the momenta to
be incoming. By Bosonic symmetries, the form
The form-factors~ $\amp_i$ can be perturbatively expanded in orders of~$\alpha_s$, 
\begin{equation}
\amp_{i} = \sum_{k=0} \left(\frac{\as}{\pi} \right) ^k \amp_i^{(k)}
\label{eq:ampexp}
\end{equation}
Where~$\amp_i^{(0)}$ and $\amp_i^{(1)}$ are the LO and NLO terms, respectively. Using Fermi's Golden Rule, we can write thee Born partonic cross-section as
\begin{equation}
\hat{\sigma}^{(0)}(\hat{s})=
\frac{\mz^2 \Gfer^2 \as(\mu_R)^2}{64 \hat{s}^2(2\pi)^3}
\int^{\hat{t}^+}_{\hat{t}^-}d\hat{t}\sum_i \left|\amp_i^{(0)}\right|^2,
\end{equation}
where
$\hat{t}^\pm=[-\hat{s}+\mh^2+\mz^2\pm\sqrt{(\hat{s}-\mh^2-\mz^2)^2-4\mh^2\mz^2}\,]/2$.
\begin{figure}
	\begin{center}
		\includegraphics[width=12cm]{./figures/Feynman_LO}
		\caption{Feynman diagrams type for the LO $gg \to Zh$ process. The triangle diagrams in a general $\xi$ gauge involve $Z$ and the neutral Goldstone~$G^0$ propagators. }
		\label{fig:dialo}
	\end{center}
\end{figure}
\par The LO has two sets of diagrams, the triangle, and box diagrams shown in~\autoref{fig:dialo}. In (a), the triangle diagrams contains a neutral Goldstone boson~$G^0$, (b) the $Z$ boson in mediated. The interplay between these two diagram types depends on the~$\xi$ gauge. Moreover, the $Z$ boson is strictly off-shell. By looking at the subamplitude~$ggZ^*$ which in the Landau gauge can be related to the decay of a massive vector boson
with mass $\sqrt{\hat{s}}$ into two massless ones, a process that is
forbidden by the Landau-Yang theorem \cite{Landau:1948kw,Yang:1950rg}.
The triangle diagrams are also proportional to the mass difference between the up and down type quarks. In this calculation, the first and second generation quarks are assumed to be massless, as well as the $b$ quark, hence light quarks loops  do not contribute to this process. The same would apply to the box diagrams (c), as they are proportional to the quark Yukawa coupling, and vanish in the massless quarks case. Moreover, triangle diagrams with $b-$ quark loops contribute to $ \sim 1\%$ of the total amplitude, when they are competed in the limit $m_b \to 0$.  
%%%%%%%
\subsection{The transverse momentum expansion}
\label{sec:ptexp}
\par Choosing to expand in small $\pt$ for the $Z$ boson, we start by expressing its transverse momentum in terms of the Mandelstam coefficients and masses 
\beq
\pt^2=\frac{\hat{t}\hat{u}-\mz^2\mh^2}{\hat{s}}.
\label{ptdef}
\eeq
From  eq.(\ref{ptdef}), together with the relation between
the Mandelstam variables, one finds 
\beq
\pt^2+\frac{\mh^2+\mz^2}{2}\leq\frac{\hat{s}}{4}+\frac{\dm^2}{\hat{s}},
\label{ptexp}
\eeq
where
$\dm = (\mh^2 -\mz^2)/2$. Eq.(\ref{ptexp}) implies 
$\pt^2/\hat{s} < 1$ that, together with the kinematical constraints
$\mh^2/\hat{s}< 1$ and
$\mz^2/\hat{s} < 1$. With these relations in mind, one can expand the amplitudes in terms of small  $\pt^2/\hat s$, $\mh^2/\hat{s}$ and $\mz^2/\hat{s}$, which is valid across $ 98\%$ of the phase space, contrary to the LME and HE limits.  The caveat for this expansion is that, the amplitude does not depend on $\pt$ explicitly. Instead, one would expand in the reduced Mandelstam variables~$t^\prime/s^\prime\ll 1$ or $u^\prime/s^\prime\ll 1$, defined as
\beq
s^\prime=p_1\cdot p_2=\frac{\hat{s}}{2},~~
t^\prime=p_1\cdot p_3=\frac{\hat{t}-\mz^2}{2},~~ u^\prime =
p_2\cdot p_3=\frac{\hat{u}-\mz^2}{2}
\eeq
and satisfy
\beq
s^\prime + t^\prime + u^\prime =\dm.
\eeq
The choice of the expansion parameter~$t^\prime$ or ~$u^\prime$  depends whether one expands in the forward or backwards kinematics. Because the process $gg \to Zh$, has two particles in the final states with different masses, the amplitude is not symmetric under the the exchange $ t^\prime \leftrightarrow u^\prime$. One therefore cannot compute the cross-section by integrating only the forward-expanded amplitude~\cite{Alasfar:2021ppe}, unlike what is done for the Higgs pair~\cite{Bonciani:2018omm}. In order to overcome this issue, one could further examine the projectors in~\autoref{app:uno} and observe that they can be split into symmetric and anti-symmetric parts w.r.t. the exchange $ t^\prime \leftrightarrow u^\prime$. Then, expand the symmetric part in the forward kinematics, like the Higgs pair case. For the anti-symmetric part, one can simply extract the antisymmetric factor by multiplying the form-factors by $1/(\hat{t}-\hat{u})$,
written as $1/( 2 s^\prime - 4 t^\prime - 2 \dm)$, then perform the expansion in the forward kinematics and finally multiply back by $(\hat{t}-\hat{u})$.

\par In order to implement the $\pt$-expansion at the Feynman diagrams level we start by splitting the momenta into longitudinal and transverse w.r.t. to the beam direction, by introducing the vector~\cite{Bonciani:2018omm}, 
\begin{align}
r^\mu &= p_1^\mu +p_3^\mu,
\end{align}
 which satisfies
\beq
r^2= \hat{t},~~ r\cdot p_1=\frac{\hat{t}-\mz^2}{2},~~
r\cdot p_2=-\frac{\hat{t}-\mh^2}{2},
\label{rsp}
\eeq
and hence can be also written as
\beq
r^\mu =-\frac{\hat{t}-\mh^2}{\hat{s}}p_1^\mu +
\frac{\hat{t}-\mz^2}{\hat{s}} p_2^\mu + r_\perp^\mu =
\frac{t^\prime}{s^\prime}\,(p_2^\mu -p_1^\mu) - \frac{\dm}{s^\prime} \, p_1^\mu +
r_\perp^\mu,
\label{rpp}
\eeq
where
\beq
r_\perp^2=-\pt^2.
\eeq
substituting the definition of~$\pt$ from eq.\eqref{ptdef} one obtains
\beq
t^\prime = -\frac{s^\prime}2 \left\{ 1 - \frac{\dm}{s^\prime} \pm
\sqrt{\left( 1 - \frac{\dm}{s^\prime} \right)^2 -
	2 \frac{\pt^2 + \mz^2}{s^\prime}} \right\}
\label{tpdef}
\eeq
that implies  that the expansion in
small $\pt$ (the minus sign case in eq.(\ref{tpdef})) can be realized
at the level of Feynman diagrams, by expanding the propagators
in terms of the vector $r^\mu$ around $r^\mu \sim 0$ or, equivalently,
$p_3^\mu \sim -p_1^\mu$, see eq.(\ref{rpp}). 

\section{Born cross-section in the $\pt$-expansion }
\label{sec:LOPtExp}
\par As a baseline test for the validity and convergence behaviour of the~$\pt$ expansion we start by computing the LO amplitude, and consequently the Born partonic cross-section in the $\pt$ expansion then compare it with the exact results found in~\cite{Kniehl:1990iva, Dicus:1988yh} . \\ We start by defining the one-loop functions appearing in the similar calculation of the Born cross-section of ~$gg \to hh$ in the same expansion made in ref.~\cite{Bonciani:2018omm}
\bea
B_0[\hat{s},\mt^2,\mt^2] \equiv  B_0^+, &
B_0[- \hat{s},\mt^2,\mt^2]  \equiv B_0^- , &\\
C_0[0,0,\hat{s},\mt^2,\mt^2,\mt^2]  \equiv  C_0^+  ,& ~~~
C_0[0,0,-\hat{s},\mt^2,\mt^2,\mt^2]  \equiv C_0^- &
\eea
\beq
B_0[q^2,m_1^2,m_2^2] = \frac1{i\pi^2}
\int \frac{d^n k}{\mu^{n-4}} \frac1{(k^2 -m_1^2)((k+q)^2-m_2^2)},
\label{Bzero}
\eeq
\begin{align}
	%	\specialcell
	{ C_0[q_a^2,q_b^2,(q_a+q_b)^2, m_1^2,m_2^2,m_3^2] = \hfill }\nn  \\
	\frac1{i\pi^2}  \int \frac{d^n k}{\mu^{n-4}} \frac1{[k^2 -m_1^2][(k+q_a)^2-m_2^2]
		[(k-q_b)^2 - m_3^2]} &
	\label{Czero}
\end{align}
are the Passarino-Veltman functions \cite{Passarino:1978jh},
with $n$ the dimension of spacetime and $\mu$ the 't Hooft mass.
%
\par There re only two non-vanishing form-factors at LO, one is symmetric~${\cal A}_2$, and the other is antisymmetric~${\cal A}_6$, in the $\pt$-expansion, these form-factors are give by, up to order ~${\mathcal O}(\pt^2)$

%\footnote{With a slight abuse of notation we indicate the
%	counting of the orders in the expansion as
%	$\mathcal{O}(\pt^{2n})$ that  actually means the inclusion of terms that
%	scale   as $(x/y)^n$,   where $x=\pt^2,\, \mz^2,\,\mh^2$ and
%	$y=\hat{s},\,\mt^2$, with respect to the $\hat{s}, \mt^2 \to \infty$
%	contribution. The latter is indicated as $\mathcal{O}(\pt^0)$ and corresponds
%	to the first non zero contribution in the expansion of the diagrams
%	in terms of the vector $r^\mu$.} 
\bea
\mathcal{A}_{2}^{(0, \triangle)} &=& -
\frac{ \pt }{\sqrt{2} \left( \mz^2+\pt^2 \right)} (\hat{s}-\dm)\,
\mt^2 C_0^+,
\label{Adt}\\
\mathcal{A}_{2}^{(0, \square)} &=&
\frac{ \pt }{\sqrt{2} \left(\mz^2+\pt^2 \right)}\, \Biggl\{ \Biggr. \nn \\
&& \Biggl( \mt^2 -\mz^2 \frac{ \hat{s}-6 \mt^2}{4 \hat{s}}-
\pt^2 \frac{ 12 \mt^4-16 \mt^2 \hat{s}+\hat{s}^2}{12 \hat{s}^2}
\Biggr)  B_0^+ \nn \\
&-& \Biggl( \mt^2 -\dm  \frac{\mt^2}{ \left( 4 \mt^2+\hat{s}\right)}
+ \mz^2 \frac{ 24 \mt^4 -6 \mt^2 \hat{s}-
	\hat{s}^2 }{4 \hat{s} \left(4 \mt^2+\hat{s}\right)} -
\nn \\
& & ~~~~~~~\pt^2 \frac{ 48 \mt^6-68 \mt^4 \hat{s}-4
	\mt^2 \hat{s}^2+\hat{s}^3 }{ 12 \hat{s}^2 \left(
	4 \mt^2 +\hat{s} \right) }  \Biggr)
B_0^- \nn \\
&+&\Biggl( 2 \mt^2- \dm +
\mz^2 \frac{3 \mt^2-\hat{s}}{\hat{s} } +
\pt^2  \frac{ 3 \mt^2 \hat{s}-2 \mt^4 }{\hat{s}^2}\Biggr)
\mt^2 \, C_0^- \nn \\
& +&\Biggl( \hat{s}-2 \mt^2 +
\mz^2  \frac{\hat{s}-3 \mt^2 }{\hat{s}}+
\pt^2 \frac{ 2 \mt^4-3 \mt^2 \hat{s}+\hat{s}^2}{\hat{s}^2 }\Biggr)
\mt^2 \, C_0^+ \nn \\
& +&\log \left(\frac{\mt^2}{\mu^2}\right) \frac{ \mt^2}{\left(4
	\mt^2+\hat{s}\right) } \Biggl( \dm + 2  \mz^2
+\pt^2 \frac{2   \hat{s}-2 \mt^2}{3 \hat{s} }\Biggr)\nn  \\
&-&\dm \frac{2 \mt^2}{\left(4 \mt^2+\hat{s}\right) } +
\mz^2  \frac{\hat{s}-12 \mt^2}{4 \left(4 \mt^2+\hat{s}\right)}
+\pt^2 \frac{ 8 \mt^4-2 \mt^2 \hat{s}+ \hat{s}^2 }
{4\hat{s} ( 4\mt^2 + \hat{s})}  \Biggl. \Biggl\},\nn \\
&&
\label{Adb}
\eea
and
\bea
\mathcal{A}_{6}^{(0, \triangle)} &=&  0,
\label{Ast} \\
\mathcal{A}_{6}^{(0, \square)} &= & 
\frac{\hat{t}-\hat{u}}{\hat{s}^2} \,\pt \Biggl[ \frac{\mt^2}2
\Bigl( B_0^- - B_0^+ \Bigr) -\frac{\hat{s}}{4} \nn \\
& -&\frac{2 \mt^2+\hat{s}}{2}\mt^2 \, C_0^- 
+\frac{2 \mt^2-\hat{s}}{2} \mt^2 \,C_0^+  \Biggr],
\label{Asb}
\eea
where these form-factors were divided into triangle ($\triangle$) and
box ($\square$) contributions, and $B_0$ functions are understood as the
finite part of the integrals on the right hand side of eq.\eqref{Bzero}.
\par Using several truncations of the $\pt$-expansion, and comparing it to the exact LO result, one can see in~\autoref{fig:LO} the exact Born partonic LO cross section (red line) as a function of the invariant mass of the $Zj$ system, $M_{Zh}$, in comparison to the $\pt$-expansions. 
%For the numerical evaluation of the cross section here and in
%the following, we used as SM input parameters
%\begin{equation}
%	\begin{split}
%		\mz=91.1876 \,\gev, ~~\mh=125.1 \gev, ~~  \mt=173.21\, \gev, \nn\\
%		m_b = 0 \, \gev, ~~G_F= 1.16637\,\gev^{-2},~~ \alpha_s(\mz)=0.118.
%	\end{split}
%\end{equation}
\begin{figure}[t]
	\centering
	\includegraphics[width=0.8\linewidth]{./figures/LO_ptexp_ratio_1000.pdf}
	\caption{The Born partonic cross-section
		as a function of the invariant mass $M_{Zh}$.
		The exact (red line) is plotted together with results at
		different orders in the $\pt$-expansion (dashed lines). In the bottom part,
		the ratio of the full result over the $\pt$-expanded one at
		various orders is shown. This plot has been already published in~\cite{Alasfar:2021ppe}}
	\label{fig:LO}
\end{figure}
From the ratio plotted in the lower panel of ~\autoref{fig:LO} , we observe that the~$\mathcal{O}(\pt^0)$ expansion is in good agreement with the exact result  when $M_{Zh}\lesssim 2\mt$. Inclusion of higher order terms up tp~$\mathcal{O}(\pt^6)$ extended the validity of the expansion to reach  $M_{Zh}\lesssim 750\gev$. This is the similar behaviour seen in ~\cite{Bonciani:2018omm} for Higgs pair.  Therefore, one would expect the $\pt$-expanded 2-loop virtual correction to be an accurate approximation with the exact (numerical) result for the region of the invariant mass of  $M_{Zh}\sim 700-750\gev$. 
Similar conclusions can be seen more explicitly in~\autoref{tab:partonic}, where it is shown that the partonic cross-section
at $\mathcal{O}(\pt^4)$ agrees with the full result for
$M_{ZH} \lesssim 600 \gev$  on the permille level 
and the agreement further improves when $\mathcal{O}(\pt^6)$ terms are included.
\begin{table}
	\renewcommand{\arraystretch}{1.2}
	\centering
	\begin{tabular}{| c| c | c | c| c| c|} \hline
		\rowcolor{lightgray}  $M_{ZH}$ [GeV]  & $\mathcal{O}(\pt^0)$ & $\mathcal{O}(\pt^2)$ & $\mathcal{O}(\pt^4)$ & $\mathcal{O}(\pt^6)$ & full \\ \hline 
		\cellcolor{lightgray} 300 & 0.3547 & 0.3393 &  0.3373 &0.3371& 0.3371 \\
		\cellcolor{lightgray} 350 & 1.9385 & 1.8413& 1.8292 &1.8279& 1.8278 \\
		\cellcolor{lightgray} 400 & 1.6990 & 1.5347 & 1.5161 &1.5143& 1.5142 \\
		\cellcolor{lightgray} 600 & 0.8328 & 0.5653 & 0.5804 &0.5792&  0.5794 \\ 
		\cellcolor{lightgray} 750 & 0.5129 & 0.2482 & 0.3129 & 0.2841 &  0.2919 \\ \hline
	\end{tabular}
	\caption{The partonic cross section $\hat{\sigma}^{(0)}$ at
		various orders in $\pt$ and the full computation for several values of $M_{ZH}$. \label{tab:partonic}}
\end{table}
%%%%
%The outcome  of the evaluation of the $gg \to ZH$ amplitude via a
%$\pt$-expansion is expressed in terms of a series of Master Integrals (MIs)
%that are functions of $\hat{s}$ and $\mt^2$ only, and whose coefficients can be
%organized in terms of powers of ratios of small  over large parameters
%where $\pt^2, \, \mh^2$ and $\mz^2$ are identified as the small parameters while
%$\mt^2$ and $\hat{s}$ as the large ones. 
%Thus, the range of validity of the expansion depends
%on  the condition that $\pt^2$ can be treated as a ``small parameter'' with
%respect to $\mt^2$ because all the other ratios, small over
%large, are always smaller than 1.

%%%%%%%
%The Feynman diagrams that contribute to the $gg \to  ZH$ amplitude up to NLO
%can be separated into triangle, box and double-triangle
%contributions, the last type appearing for the first time at the
%NLO level. Examples of LO (NLO) triangle and box
%categories are shown in fig.\ref{fig:dia} $(a)$ - $(c)$
%($(d)$ - $(f)$).
%
%whose treatment in dimensional regularization is, as well known, delicate
%and will be discussed in section \ref{sec:quattro}.
%
%In our calculation we treat all the quarks but the top as massless.
%As a consequence, the contribution to the amplitude of the first two generations
%vanishes. Concerning the third generation, the contribution of the bottom
%is present  in the triangle diagrams with the exchange of a $Z$ boson
%(fig.\ref{fig:dia}$(b),(e)$) and in the double-triangle diagrams
%(fig.\ref{fig:dia}$(g)$).
%A nice observation in ref.\cite{Altenkamp:2012sx} allows to compute
%easily the full (top+bottom) triangle contribution. As noticed in that
%reference,
%the triangle contribution with a $Z$ exchange contains a $ggZ^*$ subamplitude
%which in the Landau gauge can be related to the decay of a massive vector boson
%with mass $\sqrt{\hat{s}}$ into two massless ones, a process that is
%forbidden by
%the Landau-Yang theorem \cite{Landau:1948kw,Yang:1950rg}. As a consequence,
%the full triangle contribution can be obtained from the top triangle diagrams
%with the exchange of the unphysical scalar $G^0$, with the propagator of the
%$G^0$ evaluated in the Landau gauge. This part of the top triangle
%diagrams  can be obtained from the decay
%amplitude of a pseudoscalar boson into two gluons which is  known in
%the literature in the full mass dependence up to NLO terms \cite{Spira:1995rr,Aglietti:2006tp}. 
%
%Given the above observation, our calculation of the NLO corrections to
%the $gg \to ZH$ amplitude focuses on the analytic evaluation of the
%double-triangle (fig.\ref{fig:dia}$(g)$) and two-loop box contributions
%(fig.\ref{fig:dia}$(f)$). The former contribution is evaluated exactly.
%The latter is evaluated via two different expansions: i) via  a LME, following
%ref.\cite{Degrassi:2010eu}, up to and including ${\cal O}(1/\mt^6)$ terms,
%which is expected to work below the $2\, \mt$ threshold; ii) via an expansion in
%terms of the $Z$ transverse momentum, following ref.\cite{Bonciani:2018omm},
%whose details are presented in the next section.
\section{ NLO calculation }
\label{sec:quattro}
The virtual 2-loop corrections to~$ gg\to Zh$ are shown in~\autoref{fig:dianlo}, which involve corrections to the triangle topology (a) and (b), corrections to the box diagrams (c) and a new topology with double triangle (d). Both 2-loop corrections to the triangles, and double triangle diagrams can be computed exactly analytically. However, the 2-loop box diagrams contain master-integrals~(MI's) that have no analytic solutions, so far. The 2-loop box diagrams will be hence computed in the $\pt$-expansion.
\begin{figure}[htpb!]
	\begin{center}
		\includegraphics[width=12cm]{./figures/Feynman_NL0}
		\caption{Feynman diagrams type for the LO $gg \to Zh$ process. The triangle diagrams in a general $\xi$ gauge involve $Z$ and the neutral Goldstone~$G^0$ propagators. }
		\label{fig:dianlo}
	\end{center}
\end{figure}
\subsubsection{Renormalisation}
The 2-loop corrections to the triangle and box diagrams contain both UV and IR divergences. The first emerges from UV divergent sub-diagrams, such as top mass renormalisation and QCD vertex correction. While the lR divergences come from massless loops. In order to remove these divergences, one introduces adequate counter-terms. 
\par  We start by the gluon wavefunction renormalisation of the incoming gluons  (external legs) such that the amplitude is renormalised by $ Z_A^{1/2}$ for each gluon.
\begin{equation}
	Z_A= 1+\as \frac{2}{3\epsilon} \left( \frac{\mu_R^2}{m_t^2} \right) ^\epsilon.
\end{equation}
 The $\bar{MS}$ scheme for the top mass renormalisation has been used, in which the bare mass is replaced by the renormalised one~$ m_0 = Z_m m$  in the propagators. This can be done with multiplying $ \delta Z_m$ with the derivative of the 1-loop form-factor with respect to the  mass., here $Z_m$ is given by
\begin{equation}
	Z_m = 1+ C_F \frac{3}{\epsilon}.
\end{equation}
For the on-shell scheme we add the finite renormalisation term
\begin{equation}
	Z^{OS}_m = 1- 2 C_F,
\end{equation}
here $C_F=(N_c^2-1)/2N_c$ is one of the two Casimir invariants of QCD along with~$ C_A=N_c$. 
The $q \bar q g$ vertex correction involves a renormalisation of the strong couplings constant $ \alpha_s$ which is done via replacing the bare constant $\alpha_s^0$ with the renormalised one, hence it becomes  $ \alpha^0_s = \frac{\mu_R^{2\epsilon}}{S_\epsilon}  \Zas \alpha_s$, where
\begin{equation}
	\Zas = 1- \frac{\alpha_s}{4 \pi} \, \frac{1}{ \epsilon}\,  \left( \beta_0-\frac{2}{3} \right) \left(\frac{\mu_R^2}{m_t^2} \right) ^\epsilon,
\end{equation}
and the constant $ \beta_0 = \frac{11}{3} C_A -\frac{2}{3}N_f$, where $N_f$ is the number of "active" flavours. In the 5-flavour scheme $N_f=5$ adopted here. 
\par The loop integrals were evaluated via dimensional regularisation in $d= 4-2\epsilon$ dimensions. Which requires some caution when $\gamma_5$ is present in the amplitude. We let $\gamma_5$ naively anti-commute with all $d$-dimensional $\gamma_\mu$'s and then correct that with the finite renormalisation constant known as \textbf{ Larin counter-term}~\cite{Larin:1993tq}
\begin{equation}
	Z_5 = 1- 2\, C_F.
\end{equation}
We write renormalised amplitude as
\begin{equation}
	\mathcal M  (\alpha_s, m, \mu_R) = Z_A \mathcal M( \alpha_s^0, m^0).
\end{equation}
Putting all the above substitutions together, we get the renormalised  2 loop form-factor:
\begin{align}
	( \mathcal A ^{2 \ell})^R &= 	\mathcal A ^{2 \ell} -	\mathcal A_{UV} ^{1 \ell}- 	\mathcal A_{UV, m} ^{1 \ell} + \mathcal A_{\text{Larin}} ^{1 \ell}   \\
	\mathcal A_{UV} ^{1 \ell} &= \asr\,\frac{\beta_0}{\epsilon} \left( \frac{\mu_R^2}{\hat s} \right) ^{ -\epsilon}.  \nonumber \\
	\mathcal A_{UV, m} ^{1 \ell} &= \asr \, \left( \frac{3}{\epsilon} -2\right) C_F \left( \frac{\mu_R^2}{\hat s} \right) ^{ -\epsilon} m^0 \partial_m \mathcal A^{1 \ell} . \nonumber \\
	\\
	\mathcal	A_{\text{Larin}} ^{1 \ell}  &= - \asr \, C_F  \mathcal A ^{1 \ell} .\nonumber
\end{align}
We use the following IR-counter-term in order to cancel the IR divergences.
\begin{equation}
	\mathcal A_{IR} ^{1 \ell}  = \frac{e^{\gamma_E \epsilon}}{\Gamma(1-\epsilon)} \, \asr \left( \frac{\beta_0}{\epsilon} + \frac{ C_A}{\epsilon^2} \right)  \, \left(  \frac{\mu_R^2}{\hat s}\right) ^{2\epsilon} \mathcal A ^{1 \ell}
\end{equation}
The1-loop form-factors, need to be expanded up to order $ \mathcal O(\epsilon^2) $, for the UV and IR counter-terms.
\subsubsection{Calculation of the exact virtual corrections}
The 1 loop form factor is given by
\begin{equation}
	\mathcal F ^{1 \ell}  = \frac{m^2}{\hat s} \, C_0(\hat s,0; m,m,m) - \frac{1}{4 \hat s},
\end{equation}
where the last term, not proportional to the mass is corresponding to  the chiral anomaly .
The 2 loop form-factor, can be decomposed in terms of the colour and $ C_F$
\begin{equation}
	\mathcal F ^{2 \ell}   = C_F \mathcal F_{CF} ^{2 \ell}  + C_A \mathcal F_{CA} ^{2 \ell}
\end{equation}
The $CA$ part contains double pole $ \mathcal O( 1/\epsilon^2) $  and a single pole $ \mathcal O( 1/\epsilon) $, both  coming from the IR divergence. Whilst the $CF$ part contains a UV divergent pole that needs to be cured via mass renormalisation. The poles do not have a dependence on the renormalisation scale $ \mu_R$, however, there is a dependence on that scale in the finite part.
\subsubsection{Calculation of the $\pt$-expanded virtual corrections}
%%%
In this section we discuss our evaluation of the three different types of
diagrams that appear in the virtual corrections to the $gg \to ZH$ amplitude
at the NLO.

The triangle contribution (fig.\ref{fig:dia}$(d),(e)$) was evaluated using the
observation of  ref.\cite{Altenkamp:2012sx}, i.e.~we adapted
the result of ref.\cite{Aglietti:2006tp} for the decay of a
pseudoscalar boson into two gluons  to our case. This contribution is evaluated
exactly and explicit expressions for the form factors are presented in
appendix \ref{app:due}. We notice that if we interpret the exact result
in terms of our counting of the expansion in $\pt$, the $\pt$-expansion of the triangle contribution stops at ${\cal O}(\pt^2)$.

Given the reducible structure of the double-triangle diagrams
(fig.\ref{fig:dia}$(g)$), an exact result for the double-triangle contribution
can be derived in terms of products of one-loop Passarino-Veltman functions
\cite{Passarino:1978jh}.     
Explicit expressions for this contribution are presented in
appendix \ref{app:due}. Although we write the amplitude using a different
tensorial structure with respect to ref.\cite{Davies:2020drs} we checked,
using the relations between the two tensorial structures reported in appendix
\ref{app:uno}, that our result is in agreement with the one presented
in ref.\cite{Hasselhuhn:2016rqt}.

The box contribution (fig.\ref{fig:dia}$(f)$) was computed evaluating
the two-loop multi-scale Feynman integrals via two different expansions:
a LME up to and including $\mathcal{O}(1/\mt^6)$ terms, and  an
expansion in the transverse momentum up to and including
${\cal O}(\pt^4)$ terms.
The former expansion was used as ``control'' expansion of the latter.
Indeed, the $\pt$-expanded result actually ``contains'' the LME one. The LME
differs from the expansion in $\pt$ by the fact that $\hat{s}$ is
treated as a small parameter with respect to $\mt^2$, and not on the same
footing as in the latter case. This implies that if the $\pt$-expanded result is
further expanded in terms of the  $\hat{s}/\mt^2$ ratio the LME result has to
be recovered. This way, we were able to reproduce, at the analytic level,
our LME result. 


We conclude this section outlining some technical details concerning our
computation.  We generated the amplitudes using \texttt{FeynArts} \cite{Hahn:2000kx} and
contracted them with the projectors as defined in appendix \ref{app:uno}
using \texttt{FeynCalc }\cite{Mertig:1990an,Shtabovenko:2016sxi} and in-house
Mathematica routines.  We used  dimensional regularization and
the rule for the contraction of two epsilon tensors written in terms of
the determinant of $n$-dimensional metric tensors. This is not a consistent
procedure and  needs to be corrected. A correction term should be added
\cite{Larin:1993tq} to the form factors computed as described
above, $\amp^{(1,ndr)}_i$, namely
\beq
\amp^{(1)}_i = \amp^{(1,ndr)}_i -\frac{\as}{\pi} C_F \amp_i^{(0)}~.
\label{eq:larin}
\eeq
In order to check eq.(\ref{eq:larin}), following ref.\cite{Degrassi:2011vq}
we bypassed the problem of the treatment of 
$\gamma_5$ in dimensional regularization computing the amplitude via
a LME working in 4 dimension, employing the Background Field Method (BFM)
\cite{Abbott:1980hw} and using as regularization scheme  the Pauli-Villars
method. This result was compared with the LME evaluation of
$\amp^{(1,ndr)}_i$, finding that the difference between the two
evaluations was indeed given by the second term on the right-hand-side of
eq.(\ref{eq:larin}).

After the contraction of the epsilon tensors the diagrams were expanded as
described in section \ref{sec:tre}. They were reduced to MIs
using \texttt{FIRE} \cite{Smirnov:2014hma} and \texttt{LiteRed} \cite{Lee:2013mka}. The
resulting MIs were exactly the same as previously found for di-Higgs
production \cite{Bonciani:2018omm}. Nearly all of them are expressed
in terms of generalised harmonic polylogarithms with the exception of
two elliptic integrals \cite{vonManteuffel:2017hms, Bonciani:2018uvv}.
The top quark mass was renormalized in the onshell scheme\footnote{Different choices
	for the renormalized top mass can be easily implemented in our calculation.}
and the IR poles were subtracted as in ref.\cite{Degrassi:2016vss}.

\section{NLO results} \label{sec:sei}
We now present our numerical results for the virtual corrections.
We have implemented our results into a \texttt{FORTRAN} programme. 
For the evaluation of the generalised harmonic polylogarithms we use 
the code \texttt{handyG} \cite{Naterop:2019xaf}, while 
the elliptic integrals are evaluated using the routines of
ref.\cite{Bonciani:2018uvv}.
In order to facilitate the comparison of our results with the ones
presented in the literature,  we define the finite part of the virtual corrections
as in
ref.\cite{Davies:2020drs}\footnote{Our definition of the matrix elements
	differs by a factor of
	$\frac{1}{\hat{s}}$ from ref.\cite{Davies:2020drs}, \textit{cf}. also
	appendix \ref{app:uno}.}
\begin{equation}
	\begin{split}
		\mathcal{V}_{fin}&=\frac{G_F^2 m_Z^2}{16}\left(\frac{\as}{\pi}\right)^2
		\left[ \sum_{i} \left|\mathcal{A}_i^{(0)} \right|^2\frac{C_A}{2}\left(\pi^2-
		\log^2\left(\frac{\mu_R^2}{\hat{s}}\right)\right)\right. \\
		& \left. +2\sum_i\text{Re}\left[\mathcal{A}_i^{(0)}\left(\mathcal{A}_i^{(1)}\right)^*\right]\right]\,
		\label{eq:vfin}
	\end{split}
\end{equation}
and in the numerical evaluation of eq.(\ref{eq:vfin}) we fixed
$\mu_R= \sqrt{\hat{s}}$.

First, both the triangle and box LME contributions to $\mathcal{A}_i^{(1)}$ up
to $\mathcal{O}(1/\mt^6)$ terms  were checked, at the analytic level, against
the results of refs.\cite{Hasselhuhn:2016rqt,Davies:2020drs} finding perfect
agreement. Then,  the $\pt$-expanded results for low $M_{ZH}$ were
confronted numerically with the LME ones, finding a  good numerical agreement.
We recall that, at the same order in the expansion, 
the $\pt$-expanded terms are more accurate than
the LME ones, although computationally more demanding.

In ref.\cite{Chen:2020gae} a numerical evaluation of eq.(\ref{eq:vfin})
was presented. In that reference the exact NLO amplitude was reduced to a
set of MIs that were evaluated numerically using the code \texttt{pySecDec}
\cite{Borowka:2017idc,Borowka:2018goh}. Table 3 of that reference presents
the numerical results %\footnote{The values in table 3 of ref.\cite{Chen:2020gae}
	%are defined as $\mathcal{V}_{fin} 4/(\as^2 \alpha^2)$.\label{footnote1}} 
for various points in the
phase space. A comparison of the four points lying within the range of
validity of our expansion is shown in table \ref{tab:comparison}
using the same inputs as ref.\cite{Chen:2020gae}.
As can be seen from the table the relative difference 
between the two results is less than half a permille.
%It should be noticed that  small differences
%on the permille level can be explained not only by the different
%approaches (exact vs. $\pt$-expanded) but also by the fact that
%in ref.\cite{Chen:2020gae} the $\mz^2/\mt^2$ and $\mh^2/\mt^2$ ratios were
%approximated by a ratio of two integer numbers.

\begin{table}
	\renewcommand{\arraystretch}{1.2}
	\centering
	\begin{tabular}{| c| c | c | c| } \hline
		\rowcolor{lightgray}  $\hat{s}/m_t^2$ & $\hat{t}/m_t^2$ &  ref.\cite{Chen:2020gae} & $\mathcal{O}(\pt^6)$  \\ \hline 
		\cellcolor{lightgray} 1.707133657190554 & \cellcolor{lightgray} -0.441203767016323 & 35.429092(6) & 35.430479 \\
		\cellcolor{lightgray} 3.876056604162662 & \cellcolor{lightgray} -1.616287256345735 & 4339.045(1) & 4340.754 \\
		\cellcolor{lightgray} 4.130574250302561 & \cellcolor{lightgray} -1.750372271104745 & 6912.361(3) & 6915.797 \\
		\cellcolor{lightgray} 4.130574250302561 & \cellcolor{lightgray} -2.595461551488002 & 6981.09(2) & 6984.20  \\ \hline
	\end{tabular}
	\caption{Comparison of $\mathcal{V}_{fin} 4/(\as^2 \alpha^2)$ with the numerical results of ref.\cite{Chen:2020gae}. \label{tab:comparison}}
\end{table}

\begin{figure}[th]
	\centering
	\includegraphics[width=0.75\textwidth]{./figures/sigma_part_virt_LMEreweighted.pdf}
	\caption{$\Delta \hat{\sigma}_{virt}$ defined by eq.\eqref{eq:deltasigma}, shown as a function of $M_{ZH}$. The various orders of the $\pt$-expansion are plotted as dashed lines, while the black and red continuous lines stand for the LME and  reweighted $m_t \rightarrow \infty$ results, respectively.}
	\label{fig:deltasigma}
\end{figure}
In order to present our results we define a virtual part of the partonic cross section
from the finite part of the virtual corrections in eq.\eqref{eq:vfin} by
\begin{equation}
	\Delta \hat{\sigma}_{virt}=
	\int_{\hat{t}^-}^{\hat{t}^+} d\hat{t}
	\frac{\alpha_s}{16\pi^2}\frac{1}{\hat{s}^2}\mathcal{V}_{fin}\,
	\label{eq:deltasigma}
\end{equation}
and  show it in fig.\ref{fig:deltasigma}. The dashed lines in the
plot show the different orders in our expansion. 
For all parts of the matrix elements we use the best results
available, i.e.~both $\mathcal{A}^{(0)}$ and the double-triangle
contribution are evaluated  exactly, while for
$\mathcal{A}^{(1)}$ we use the various orders in the $\pt$-expansion.
For comparison, we show the results where
$\mathcal{A}^{(1)}$ is replaced by the one computed in LME up to
$\mathcal{O}(1/m_t^6)$ (full black line), which as mentioned before is valid
up to $M_{ZH}< 2 \mt$. We see that within the validity of the LME our
results agree well with it.
Furthermore, we show the results in the infinite top
mass limit reweighted by the full amplitudes squared (full red line), corresponding to the
approach of ref.\cite{Altenkamp:2012sx}, keeping though the double triangle
contribution in full top mass dependence. 
Differently from the LME line, the $\mt \to \infty$ reweighted one
shows a behaviour, for  $M_{ZH} \gtrsim 400\gev$, similar to the behaviour of
the $\pt$ lines. Still,   the difference
between the reweighted result and the $\pt$-expanded ones is  significant.
The $\pt$-expanded results show
very good convergence.  The zero order in our expansion agrees
extremely well with the higher orders in the expansion, and all the
three results are very close up to $M_{ZH} \sim 500\gev$.


Finally, we note that the evaluation of $\mathcal{V}_{fin}$ requires a
running time per phase space point less than
one second. In addition, the integration over the $\hat{t}$ variable
in eq.(\ref{eq:deltasigma}) converges very well, such that 
fig.\ref{fig:deltasigma} could be produced on a standard laptop in a
few hours. Thus, our computation of the two-loop virtual corrections
in $gg \to ZH$ is suitable to be used within a Monte Carlo code.


\section{Conclusion \label{sec:conclusion}}
In this paper, we computed  the two-loop NLO virtual corrections to the 
$gg \to ZH$ process. Among the two-loop Feynman diagrams contributing
to the process,  the ones belonging to the triangle and
double-triangle topology were computed exactly. The ones  belonging 
to the box topology, which contain multiscale integrals, were evaluated via an
expansion in the $Z$ transverse momentum. This novel approach of
computing a process in the forward kinematics
was originally proposed in ref.\cite{Bonciani:2018omm} for
double Higgs production where the particles in the final state have
the same mass. In this paper,  we  extended
the method to the more general case of two different masses in the
final state and to a process whose amplitude is not symmetric
under the  $\hat{t}\leftrightarrow \hat{u}$ exchange.

The result of the evaluation of the box contribution is expressed,
both at one- and two-loop level, in terms of the
same set of MIs that was found in ref.\cite{Bonciani:2018omm} for
double Higgs production.  The two-loop MIs  can be all
expressed in terms of generalised harmonic polylogarithms with the
exception of two elliptic integrals.

As we have shown explicitly at the LO, the range of validity of our
computation covers values of the invariant mass
$M_{ZH}\lesssim 750\text{ GeV}$ corresponding to 98.5\% of the phase
space at LHC energies.  We showed that few terms in our
expansion were sufficient to obtain an incredible good agreement with the
numerical evalution of ${\mathcal V}_{fin}$ presented in ref.\cite{Chen:2020gae},
at the level of a permille or less difference between our analytic result and the numerical one.

The advantage of our analytic approach compared to the numerical
calculation is also in the computing time. With an average evaluation
time of half a second per phase space point, an inclusion into a Monte Carlo
programme is realistic. Due to the flexibility of our analytic
results, an application to beyond-the-Standard Model is certainly
possible.

Finally, we remark that our calculation complements
nicely the results obtained in ref.\cite{Davies:2020drs} using a high-energy
expansion, that according to the authors provides precise results for
$\pt \gtrsim 200\gev$. The merging of the two analyses is going to provide
a result that covers the whole phase space, can be easily implemented into a
Monte Carlo code and  presents the flexibility of an analytic calculation.
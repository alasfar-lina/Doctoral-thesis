%!TEX encoding = UTF-8 Unicode
% !TeX spellcheck = en_GB
\chapter{Conclusion}
Constraints on Higgs observables are deeply intertwined with the top quark and flavour physics. This has been highlighted throughout the entirety of this thesis and the literature reviewed within. \\  The era of Higgs precision measurements is on the horizon, prompting the inclusion of higher-order corrections to Higgs processes, which requires improved techniques for their calculation. An example of these techniques is the $\pt$ expansion, first employed to obtain an analytic form for the Higgs pair virtual corrections~\cite{Bonciani:2018omm}. This technique was used in~\autoref{chap:hz} for obtaining the QCD two-loop corrections of the gluon fusion component of $Zh$, which constitutes the main source of theoretical uncertainty on the associated production of the Higgs boson with a $Z$ boson. The true power of this method is seen when combined with Pad\'e approximants to bridge it with other expansions to obtain an analytic form for the virtual corrections covering the entire invariant mass spectrum~\cite{Bellafronte:2022jmo}. \\ The use of higher-order calculations in SMEFT opened the potential for probing the Higgs trilinear self-coupling~\cite{Gorbahn:2016uoy, Degrassi:2016wml, Bizon:2016wgr, Maltoni:2017ims, Degrassi:2021uik}, and show connections between four-top operators and EWPO~\cite{Dawson:2022bxd}. The nexus between the SMEFT four-heavy quark operators and the Higgs self-coupling is explored in~\autoref{chap:4topSingleHiggs}, via the inclusion of NLO SMEFT effects in single-Higgs rates. \\ Precision Higgs measurements will not be complete without observing Higgs pair production, the aspired jewel process of the HL-LHC, which carries the most potential for measuring the elusive Higgs self-coupling,  consequently revealing the shape of the Higgs potential.\\ In~\autoref{chap:lightyuk}, I have demonstrated the potential of this process in constraining other ``difficult'' Higgs observable; its interaction with light quarks. Higgs pair production is treated as a multivariate problem and aspects of interpretable machine learning were employed to increase the selection efficiency~\cite{Grojean:2020ech}. Using a BDT-classifier interfaced with Shapley values as an interpretability layer, it was possible to constrain the trilinear coupling along with the up-and down-quark Yukawa coupling enhancements within SMEFT. The interpretability allowed for an optimised classifier and added physics understanding and confidence. The constraints projected for HL-LHC on up-quark Yukawa coupling enhancement obtained from this analysis are the most stringent amongst all other probes~\cite{Soreq:2016rae,Falkowski:2020znk,Aguilar-Saavedra:2020rgo,Yu:2017vul}, and even the global analysis~\cite{deBlas:2019rxi}.  \\ I discussed in~\autoref{chap:flav} the recent flavour anomalies within the SMEFT framework. When these anomalies are confronted with EWPO, a marked tension of up to $ 6\sigma$ is observed between the data from $B$ decays and EWPO, further highlighting the interplay between these anomalies and EWPO~\cite{Bhattacharya:2014wla,Feruglio:2016gvd,Celis:2017doq,Buttazzo:2017ixm, Kumar:2018kmr,Ciuchini:2019usw,Aebischer:2019mlg,Cornella:2019hct}.\\ This conundrum can be resolved by a global fit involving EW and flavour data. A minimalist SMEFT model, assuming no new flavour spurions are involved, would generate LUV at the loop level and involves semi-leptonics operators from the top and Higgs sectors, namely and 	$C_{L/eu}$ and  $ C_{\phi L/e}$. I have then showcased UV-complete models ascertained from this fit to explain these flavour anomalies based on a top-phillic $Z^\prime$ and leptoquarks.
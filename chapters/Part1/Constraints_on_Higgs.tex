%!TEX encoding = UTF-8 Unicode
% !TeX spellcheck = en_GB
%%%%%%%%%%%%%%%%%%%%%%%%%%%%%%%%%%%%%%
\chapter{Constraints on the Higgs properties }\label{chap:HiggsConstr}
%%%%%%%%%%%%%%%%%%%%%%%%%%%%%%%%%%%%%%
In this chapter, the bounds on the Higgs sector will be discussed. Starting from an overview of the theoretical constraints on the Higgs potential, like the quantum triviality and unitarity. Then, the state-of-the-art experimental results on Higgs properties and couplings measurements will be discussed. However, despite many of the Higgs boson properties have been measured with good accuracy, there are still difficult observables in the Higgs sector and some open problems. These will be addressed at the end of this chapter.   
%%
\section{Theoretical constraints}


\section{Experimental}
We also provide in this appendix the experimental measurements of the signal strengths at the LHC Run II and the CMS projections for the HL-LHC (scenario S2, see \cite{Cepeda:2019klc}) that we used in the fits in this paper. These inputs are summarised in table~\ref{table:resHiggsExp}.
\newpage
\begingroup
\thispagestyle{plain}
\begin{table}[htb!]
\centering
\vspace{-1 cm}
 \footnotesize{ 
	{\renewcommand{\arraystretch}{0.75 }%
\begin{tabular}{clccc}
\toprule
\toprule
\multirow{5}{*}{ {\normalsize Production}}  &\multirow{5}{*}{ {\normalsize Decay}}&\multicolumn{2}{c}{ $\mu_{\mathrm{Exp}} \pm \delta \mu_{\mathrm{Exp}}$  (symmetrised)} &\multirow{5}{*}{ {\normalsize Ref.}} \\
%\cmidrule(r){3-4}
&   & { \bf     \scriptsize           LHC Run-II}&{ \bf  \scriptsize HL-LHC}&   \\
\cmidrule(r){3-4}
&   & { \scriptsize                   CMS $137 \, \mathrm{fb}^{-1} $}&  \multirow{2}{*}{CMS $3 \, \mathrm{ab}^{-1}$}&   \\
&   &  { \CG \scriptsize                   ATLAS $139 \, \mathrm{fb}^{-1} $} & &  \\
\midrule
\midrule
\multirow{ 13}{*}{ \normalsize ggF}         & \multirow{2}{*}{$h\to \gamma  \gamma$} & { \scriptsize                  $0.99 \pm 0.12$}& \multirow{2}{*}{$1.000\pm 0.042$}& \multirow{2}{*}{\cite{ATLAS:2020qdt,CMS:2021kom,CMS-PAS-FTR-18-011}}\\
                                           &                                                          &{ \scriptsize                   \CG $1.030 \pm 0.110$}&& \\ 
                                           \cmidrule(r){2-5}
                                           %%%%%%
                                    &  \mr{$h\to Z Z^*$}          & { \scriptsize                  $0.985 \pm 0.115$}&\multirow{2}{*}{$1.000 \pm 0.040$}&\multirow{7}{*}{\cite{ATLAS:2020qdt,CMS:2020gsy,CMS-PAS-FTR-18-011}}  \\
                                     &                                                      &{ \scriptsize                   \CG $0.945 \pm 0.105$}&& \\
                                     \cmidrule(r){2-4}
                                       %%%%%%
                                    &\mr{ $h\to W W^*$}         & { \scriptsize                  $1.285 \pm 0.195$} &\mr{ $1.000 \pm 0.037$} &\\
                                    & &                                            { \scriptsize                   \CG$1.085 \pm 0.185$} & &\\
                                                                         \cmidrule(r){2-4}
                                     %%%%%%
                                    &\mr{ $h\to \tau^+\tau^- $ }         & { \scriptsize                  $0.385 \pm 0.385$} &\mr{ $1.000 \pm 0.055$} &\\
                                 & &                                            { \scriptsize                   \CG$1.045 \pm 0.575$} & &\\
                                 \cmidrule(r){2-5}
                                 %%%%%%

                                  &\mr{ $h\to  b \bar b$  }      & { \scriptsize                 $2.54 \pm 2.44$} &\mr{ $1.000 \pm 0.247$} &\mr{\cite{CMS:2020gsy,CMS-PAS-FTR-18-011}}\\
                               & &                                            { \scriptsize                   \CG--} & &\\
                                 \cmidrule(r){2-5}
                               %%%%%%  %%%%%%   %%%%%%
                                  &\mr{ $h\to  \mu^+ \mu^-$  }      & { \scriptsize      $0.315 \pm 1.815$} &\mr{ $1.000 \pm 0.138$} &\mr{\cite{CMS:2020gsy,CMS-PAS-FTR-18-011} }\\
& &                                            { \scriptsize                   \CG--} & &\\
%%%%%%  %%%%%%   %%%%%%                               
\midrule
\midrule
%\crowcolor
\multirow{13}{*}{ \normalsize VBF}      
                                     %%%%%%
										&\mr{ $h\to \gamma  \gamma$ }         & { \scriptsize                  $1.175 \pm 0.335$ } &\mr{ $1.000 \pm 0.128$} & \mr{\cite{ATLAS:2020qdt,CMS:2021kom,CMS-PAS-FTR-18-011}}\\
										& &                                           { \scriptsize                   \CG$1.325 \pm 0.245$} & &\\
\cmidrule(r){2-5}
%%%%%%                                   
                                     &\mr{$h\to Z Z^*$ }         & { \scriptsize                  $0.62 \pm 0.41$ } &\mr{ $1.000 \pm 0.134$} & \multirow{7}{*}{\cite{ATLAS:2020qdt,CMS:2020gsy,CMS-PAS-FTR-18-011}}\\
                                    & &                                            { \scriptsize                   \CG$1.295 \pm 0.455$} & &\\
                                                                                             \cmidrule(r){2-4}
%%%%%%

                                   &\mr{$h\to W W^*$}         & { \scriptsize                  $0.65 \pm 0.63$ } &\mr{ $1.000 \pm 0.073$} & \\
                                    & &                                            { \scriptsize                   \CG$0.61 \pm 0.35$} & &\\
 \cmidrule(r){2-4}
%%%%%%0
                                   &\mr{$h\to \tau^+\tau^- $}         & { \scriptsize                  $1.055 \pm 0.295$ } &\mr{ $1.000 \pm 0.044$} & \\
& &                                            { \scriptsize                   \CG$1.17 \pm 0.55$} & &\\
\cmidrule(r){2-5}
%%%%%%0                                    
                                    &\mr{$h\to  b \bar b$}         & { \scriptsize                   -- } &\mr{--} & \mr{\cite{ATLAS:2020qdt} }\\
                                    & &                                            { \scriptsize                   \CG$3.055 \pm 1.645$} & &\\
                                    
                                  \cmidrule(r){2-5}
 %%%%%%  %%%%%%   %%%%%%
 &\mr{ $h\to  \mu^+ \mu^-$  }      & { \scriptsize               $3.325 \pm 8.075$} &\mr{ $1.000 \pm 0.540$} &\mr{ \cite{CMS-PAS-FTR-18-011}}\\
 & &                                            { \scriptsize                   \CG--} & & \\                                   
\midrule
\midrule
\multirow{10}{*}{ \normalsize  $t\bar t h$} 
%%%%%%0                                    
&\mr{ $h\to \gamma  \gamma$}         & { \scriptsize                $1.43 \pm 0.30$ } &\mr{$1.000 \pm 0.094$} & \mr{ \cite{ATLAS:2020qdt,CMS:2021kom,CMS-PAS-FTR-18-011} }\\
& &                                            { \scriptsize                   \CG$0.915 \pm 0.255$} & &\\

\cmidrule(r){2-5}

                                    
%%%%%%0                                    
&\multirow{3}{*} { $h\to V V^*$   }         & { \scriptsize              $0.64 \pm 0.64$({\color{Mahogany}$ZZ^*$}) } &{ \scriptsize   $1.000 \pm 0.246$ ({\color{Mahogany}$ZZ^*$}) } & \multirow{8}{*}{\cite{ATLAS:2020qdt,CMS:2020gsy,CMS-PAS-FTR-18-011}}  \\
& &                                            { \scriptsize                   $0.945\pm 0.465$ ({\color{Mahogany} $W W^*$})} & { \scriptsize   $1.000 \pm 0.097$ ({\color{Mahogany} $W W^*$})} &\\
& &                                            { \scriptsize                   \CG $1.735 \pm 0.545$} & { \scriptsize   --}&\\
\cmidrule(r){2-4}                                    

&\mr{$h\to \tau^+\tau^- $}         & { \scriptsize                $0.845 \pm 0.705$} &\mr{ $1.000 \pm 0.149$} & \\
& &                                            { \scriptsize                   \CG $1.27 \pm 1.0$} & &\\
\cmidrule(r){2-4}                                    

&\mr{ $h\to  b \bar b$  }         & { \scriptsize                 $1.145 \pm 0.315$} &\mr{ $1.000 \pm 0.116$} & \\
& &                                            { \scriptsize                   \CG $0.795 \pm 0.595$} & &\\                                                        
\midrule
\midrule
\multirow{9}{*}{ \normalsize $Vh$}        

                      
&\mr{ $h\to \gamma  \gamma$  }         & { \scriptsize   $0.725 \pm 0.295$ } &{ \scriptsize   $1.000 \pm 0.233$ ({\color{Mahogany}$Zh$}) } & \multirow{2}{*}{ \cite{ATLAS:2020qdt,CMS:2021kom,CMS-PAS-FTR-18-011}  }  \\
& &                                            { \scriptsize                   \CG $1.335 \pm 0.315$} & { \scriptsize   $1.000 \pm 0.139$ ({\color{Mahogany} $W^\pm h$})} &\\
\cmidrule(r){2-5}           
%%%%%%0              
                                    
&\mr{ $h\to Z Z^*$    }         & { \scriptsize   $1.21 \pm 0.85$ } &{ \scriptsize   $1.000 \pm 0.786$ ({\color{Mahogany}$Zh$}) } & \multirow{2}{*}{ \cite{ATLAS:2020qdt,CMS:2020gsy,CMS-PAS-FTR-18-011}  }  \\
& &                                            { \scriptsize                   \CG $1.635 \pm 1.025$} & { \scriptsize   $1.000 \pm 0.478$ ({\color{Mahogany} $W^\pm h$})} &\\
\cmidrule(r){2-5}           
%%%%%%0                                         
                                  
 &\mr{ $h\to W W^*$    }         & { \scriptsize   $1.850\pm 0.438$ } &{ \scriptsize   $1.000 \pm 0.184$ ({\color{Mahogany}$Zh$}) } & \multirow{2}{*}{  \cite{CMS:2021ixs,CMS-PAS-FTR-18-011} }  \\
 & &                                            { \scriptsize                   \CG --} & { \scriptsize   $1.000 \pm 0.138$ ({\color{Mahogany} $W^\pm h$})} &\\
 \cmidrule(r){2-5}           
 %%%%%%0                                                                                                            
 &\mr{$h\to  b \bar b$      }         & { \scriptsize  -- } &{ \scriptsize   $1.000 \pm 0.065$ ({\color{Mahogany}$Zh$}) } & \multirow{2}{*}{  \cite{ATLAS:2020qdt,CMS-PAS-FTR-18-011} }  \\
& &                                            { \scriptsize                   \CG $1.025 \pm 0.175$} & { \scriptsize   $1.000 \pm 0.094$ ({\color{Mahogany} $W^\pm h$})} &\\

%%%%%%0                                        
                                    
\midrule
\midrule
\multirow{2}{*}{ \normalsize $Zh$ { \scriptsize {\color{Mahogany} CMS     }   }}    & $h\to \tau^+\tau^- $ & $1.645 \pm 1.485$&\multirow{5}{*}{--} &\multirow{5}{*}{ \cite{CMS:2020gsy} }  \\
& $h\to  b \bar b$       &$0.94 \pm 0.32$&&\\                         
 \cmidrule(r){2-3}    
\multirow{2}{*}{ \normalsize $W^\pm h${ \scriptsize {\color{Mahogany} CMS     }   }}           & $h\to \tau^+\tau^- $ &$3.08 \pm 1.58$&&\\
& $h\to  b \bar b$      & $1.28 \pm 0.41$&&\\                  
\midrule
\midrule
\end{tabular}
}
}
\caption{The experimental single Higgs production and decay rates measurements from the  complete  data of LHC Run II and projections for the HL-LHC. The uncertainties were symmetrised here. The table is published in~\cite{Alasfar:2022zyr}.  }
\label{table:resHiggsExp}
\end{table} 
\endgroup
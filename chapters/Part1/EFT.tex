%!TEX encoding = UTF-8 Unicode
% !TeX spellcheck = en_GB
%%%%%%%%%%%%%%%%%%%%%%%%%%%%%%%%%%%%%%
\chapter{Higgs and effective field theories }\label{chap:HiggsEFT}
%%%%%%%%%%%%%%%%%%%%%%%%%%%%%%%%%%%%%%
\par If the new BSM degrees of freedom are much heavier than the electroweak scale, a general description of potential new physics effects can be formulated in the language of an effective field theory (EFT). One possibility of such a parameterization is the so-called Standard Model EFT (SMEFT), in which new physics effects are given in terms of higher-dimensional operators involving only SM fields and that also respect the SM gauge symmetries.  The dominant effects on Higgs physics, electroweak physics and top quark physics stem from dimension-six operators, suppressed by the new physics scale $\Lambda$. This approach is justified in the limit 
in which energy scales $E\ll \Lambda$ are probed. 
 \par In the presence of a gap between the electroweak scale and the scale of new physics, $\Lambda$, the effect of new particles below the new physics scale can be described by an EFT. In the case of the SMEFT, the SM Lagrangian is extended by a tower of higher-dimensional operators, ${\cal O}_i$, built using the SM symmetries and fields (with the Higgs field belonging to an $SU(2)_L$ doublet), and whose interaction strength is controlled by Wilson coefficients, $C_i$, suppressed by the corresponding inverse power of $\Lambda$. In a theory where baryon and lepton number are preserved, the leading order (LO) new physics effects are described by the dimension-six  SMEFT Lagrangian,
%
\begin{equation}
	\mathcal{L}_{\mathrm{SMEFT}}^{d=6}=\mathcal{L}_{\SM} + \frac{1}{\Lambda^2}\sum_i C_i  {\cal O}_i.
\end{equation}
A complete basis of independent dimension-six operators was presented for the first time in \cite{Grzadkowski:2010es}, the so-called \textit{ Warsaw basis}.
%!TEX encoding = UTF-8 Unicode
% !TeX spellcheck = en_GB
%%%%%%%%%%%%%%%%%%%%%%%%%%%%%%%%%%%%%%
\chapter{Higgs and effective field theories }\label{chap:HiggsEFT}
%%%%%%%%%%%%%%%%%%%%%%%%%%%%%%%%%%%%%%
\par The Standard Model (SM) has been concluded after the Higgs boson discovery~\cite{ATLAS:2012yve,CMS:2012qbp}, followed by its extensive characterisation by the ATLAS and CMS experiments including its general properties~\cite{ATLAS:2015yey,ATLAS:2018tdk,CMS:2017dib,CMS:2020xrn,ATLAS:2018jym,CMS:2019ekd,ATLAS:2015zhl, CMS:2014nkk},  cross-sections~\cite{CMS:2018gwt,ATLAS:2019jst,ATLAS:2019mju,CMS:2019chr} and couplings to electroweak and heavier fermions~\cite{ATLAS2021vrm,CMS:2020gsy}. Nonetheless, there are many open questions regarding the nature of the Higgs boson, which are left unanswered. This includes the shape of Higgs boson potential, its coupling to light quarks and the hierarchy problem. Answering these questions opens space for extending the SM by New Physics (NP) degrees of freedom.
\par In order to make the search for NP more accessible and model-agnostic, we revert to~\textbf{effective field theories}~(EFT), one of the most perspicacious concepts of quantum field theory. In the EFT framework, the interactions mediated by NP at the small scale of arbitrary complexity can be systematically simplified by approximating these interactions via integrating the UV degrees of freedom, leaving numerable operators consisting of higher dimensional operator consisting of SM fields, which are added to the SM.  
\par These ``phenomenological Lagrangians'', as called by Weinberg~\cite{WEINBERG1979327}, are not necessarily renormalisable but still allow for robust predictions that can be tested at colliders, including higher-order effects. \\
This chapter is organised as follows:  In ~\autoref{sec:smeft}, the Higgs sector of Standard Model effective field theory~(SMEFT) is presented along with the parametrisation of single and di-Higgs rates in terms of the SMEFT Wilson coefficients. In contrast to the SMEFT formalism, \autoref{sec:chiral} will present a non-linear EFT formalism known as the EW Chiral Lagrangian~(EWChL) or the  Higgs effective field theory (HEFT). Finally, I will conclude this chapter in~\autoref{sec:concefts}.
%%%%%%%%%
\section{The Higgs boson and Standard Model effective field theory \label{sec:smeft}}
\par  The idea behind the Standard Model effective field theory is to preserve the SM symmetries and fields. In particular, the Higgs boson $h(x)$ is assumed to originate from the doublet $\phi$, like the SM.  New operators of higher mass dimension are added to dimension-four SM operators. These new operators consist of the SM fields and obey its symmetries. Although these operators are not renormalisable, they are, nonetheless, predictive.
 \par From simple dimensional analysis, it is known that higher dimensional operators need to contain an inverse mass with some power $p=4-d$ in the couplings. Therefore, it is not needed to use the infinite number of the Wilson coefficients~$C_i$ when fitting to experimental measurements. Since, the higher dimensional operators are suppressed by higher powers of the UV scale $\Lambda$, hence their effect can be neglected. For example, if the NP scale is set to  $\Lambda =1$, then the effects of dimension-six operators will be at the per cent level. At the same time, dimension-eight operators will have effects of order~$\sim10^{-4}$, allowing to ignore the dimension-eight and higher operators in the majority of the LHC studies.  Regarding dimension-five, there is only one operator called the Weinberg operator~\cite{PhysRevLett.43.1566}, which does not have a considerable Higgs phenomenology. Hence, I shall be discussing SMEFT with dimension-six operators only as  they have the most prominent collider phenomenology~\cite{BUCHMULLER1986621,Hagiwara:1993ck}, for studies on Higher-dimensional SMEFT operators cf.~\cite{Lehman:2014jma,Lehman:2015coa,Henning:2015alf,Aguilar-Saavedra:2010uur}. The SMEFT Lagrangian up to dimension-six operators is given by
%%%%%%%%%%
\begin{equation}
	\mathcal{L}_{\mathrm{SMEFT}}^{d=6}=\mathcal{L}_{\SM} + \frac{1}{\Lambda^2}\sum_i C_i  {\cal O}_i.
	\label{smeftdim6}
\end{equation}
%%%%%%%%%%%
\begin{table}
	\begin{center}
		\footnotesize
			\vspace{-.35cm}
		\hspace{-2.7 cm}
		\begin{minipage}[t]{4.6cm}
			\renewcommand{\arraystretch}{1.5}
			\begin{tabular}[t]{c|c}
				\multicolumn{2}{c}{$X^3$} \\
				\toplinetwo
				$\mathcal{O}_G$                & $f^{ABC} G_\mu^{A\nu} G_\nu^{B\rho} G_\rho^{C\mu} $ \\
				%
				$\mathcal{O}_{\widetilde G}$          & $f^{ABC} \widetilde G_\mu^{A\nu} G_\nu^{B\rho} G_\rho^{C\mu} $ \\
				%
				$\mathcal{O}_W$                & $\epsilon^{IJK} W_\mu^{I\nu} W_\nu^{J\rho} W_\rho^{K\mu}$ \\ 
				%
				$\mathcal{O}_{\widetilde W}$          & $\epsilon^{IJK} \widetilde W_\mu^{I\nu} W_\nu^{J\rho} W_\rho^{K\mu}$ \\
			\end{tabular}
		\end{minipage}
		%
		%
		%
		%
		\begin{minipage}[t]{4.6cm}
			\renewcommand{\arraystretch}{1.5}
			\begin{tabular}[t]{c|c}
				\multicolumn{2}{c}{Pure Higgs} \\
				\toplinetwo
				$\mathcal{O}_{\phi\Box}$ & $(\phi^\dag \phi)\Box(\phi^\dag \phi)$ \\
				%
				$\mathcal{O}_{\phi D}$   & $\ \left(\phi^\dag D_\mu \phi\right)^* \left(\phi^\dag D_\mu \phi\right)$ \\
				%
				$\mathcal{O}_\phi$       & $(\phi^\dag \phi)^3$ 
			\end{tabular}
		\end{minipage}
		%
		%
		\begin{minipage}[t]{2.7cm}
			
			\renewcommand{\arraystretch}{1.5}
			\begin{tabular}[t]{c|c}
				\multicolumn{2}{c}{$ \psi^2\phi^3 + \hbox{h.c.}$} \\
				\toplinetwo
				$\mathcal{O}_{e\phi}$           & $(\phi^\dag \phi)(\bar l_p e_r \phi)$ \\
				%
				$\mathcal{O}_{u\phi}$          & $(\phi^\dag \phi)(\bar q_p u_r \widetilde \phi )$ \\
				%
				$\mathcal{O}_{d\phi}$           & $(\phi^\dag \phi)(\bar q_p d_r \phi)$\\
			\end{tabular}
		\end{minipage}
		
		\vspace{0.25cm}
				\hspace{-2.7 cm}
		\begin{minipage}[t]{4.6cm}
			\renewcommand{\arraystretch}{1.5}
			\begin{tabular}[t]{c|c}
				\multicolumn{2}{c}{$X^2\phi^2$} \\
				\toplinetwo
				$\mathcal{O}_{\phi G}$     & $\phi^\dag \phi\, G^A_{\mu\nu} G^{A\mu\nu}$ \\
				%
				$\mathcal{O}_{\phi\widetilde G}$         & $\phi^\dag \phi\, \widetilde G^A_{\mu\nu} G^{A\mu\nu}$ \\
				%
				$\mathcal{O}_{\phi W}$     & $\phi^\dag \phi\, W^I_{\mu\nu} W^{I\mu\nu}$ \\
				%
				$\mathcal{O}_{\phi\widetilde W}$         & $\phi^\dag \phi\, \widetilde W^I_{\mu\nu} W^{I\mu\nu}$ \\
				%
				$\mathcal{O}_{\phi B}$     & $ \phi^\dag \phi\, B_{\mu\nu} B^{\mu\nu}$ \\
				%
				$\mathcal{O}_{\phi\widetilde B}$         & $\phi^\dag \phi\, \widetilde B_{\mu\nu} B^{\mu\nu}$ \\
				%
				$\mathcal{O}_{\phi WB}$     & $ \phi^\dag \tau^I \phi\, W^I_{\mu\nu} B^{\mu\nu}$ \\
				%
				$\mathcal{O}_{\phi\widetilde W B}$         & $\phi^\dag \tau^I \phi\, \widetilde W^I_{\mu\nu} B^{\mu\nu}$ 
			\end{tabular}
		\end{minipage}
		%
		%
		\begin{minipage}[t]{4.6cm}
			\renewcommand{\arraystretch}{1.5}
			\begin{tabular}[t]{c|c}
				\multicolumn{2}{c}{$\psi^2 X\phi+\hbox{h.c.}$} \\
				\toplinetwo
				$\mathcal{O}_{eW}$      & $(\bar l_p \sigma^{\mu\nu} e_r) \tau^I \phi W_{\mu\nu}^I$ \\
				%
				$\mathcal{O}_{eB}$        & $(\bar l_p \sigma^{\mu\nu} e_r) \phi B_{\mu\nu}$ \\
				%
				$\mathcal{O}_{uG}$        & $(\bar q_p \sigma^{\mu\nu} T^A u_r) \widetilde \phi \, G_{\mu\nu}^A$ \\
				%
				$\mathcal{O}_{uW}$        & $(\bar q_p \sigma^{\mu\nu} u_r) \tau^I \widetilde \phi \, W_{\mu\nu}^I$ \\
				%
				$\mathcal{O}_{uB}$        & $(\bar q_p \sigma^{\mu\nu} u_r) \widetilde \phi \, B_{\mu\nu}$ \\
				%
				$\mathcal{O}_{dG}$        & $(\bar q_p \sigma^{\mu\nu} T^A d_r) \phi\, G_{\mu\nu}^A$ \\
				%
				$\mathcal{O}_{dW}$         & $(\bar q_p \sigma^{\mu\nu} d_r) \tau^I \phi\, W_{\mu\nu}^I$ \\
				%
				$\mathcal{O}_{dB}$        & $(\bar q_p \sigma^{\mu\nu} d_r) \phi\, B_{\mu\nu}$ 
			\end{tabular}
		\end{minipage}
		%
		%
		\begin{minipage}[t]{2.7cm}
			\renewcommand{\arraystretch}{1.5}
			\begin{tabular}[t]{c|c}
				\multicolumn{2}{c}{$\psi^2\phi^2 D$} \\
				\toplinetwo
				$\mathcal{O}_{\phi l}^{(1)}$      & $(\phi^\dag i\overleftrightarrow{D}_\mu \phi)(\bar l_p \gamma^\mu l_r)$\\
				%
				$\mathcal{O}_{\phi l}^{(3)}$      & $(\phi^\dag i\overleftrightarrow{D}^I_\mu \phi)(\bar l_p \tau^I \gamma^\mu l_r)$\\
				%
				$\mathcal{O}_{\phi e}$            & $(\phi^\dag i\overleftrightarrow{D}_\mu \phi)(\bar e_p \gamma^\mu e_r)$\\
				%
				$\mathcal{O}_{\phi q}^{(1)}$      & $(\phi^\dag i\overleftrightarrow{D}_\mu \phi)(\bar q_p \gamma^\mu q_r)$\\
				%
				$\mathcal{O}_{\phi q}^{(3)}$      & $(\phi^\dag i\overleftrightarrow{D}^I_\mu \phi)(\bar q_p \tau^I \gamma^\mu q_r)$\\
				%
				$\mathcal{O}_{\phi u}$            & $(\phi^\dag i\overleftrightarrow{D}_\mu \phi)(\bar u_p \gamma^\mu u_r)$\\
				%
				$\mathcal{O}_{\phi d}$            & $(\phi^\dag i\overleftrightarrow{D}_\mu \phi)(\bar d_p \gamma^\mu d_r)$\\
				%
				$\mathcal{O}_{\phi u d}$ + h.c.   & $i(\widetilde \phi ^\dag D_\mu \phi)(\bar u_p \gamma^\mu d_r)$\\
			\end{tabular}
		\end{minipage}
		
		\vspace{0.25cm}
\hspace{-2.7 cm}
		
		\begin{minipage}[t]{4.95cm}
			\renewcommand{\arraystretch}{1.5}
			\begin{tabular}[t]{c|c}
				\multicolumn{2}{c}{$(\bar LL)(\bar LL)$} \\
				\toplinetwo
				$\mathcal{O}_{ll}$        & $(\bar l_p \gamma_\mu l_r)(\bar l_s \gamma^\mu l_t)$ \\
				%
				$\mathcal{O}_{qq}^{(1)}$  & $(\bar q_p \gamma_\mu q_r)(\bar q_s \gamma^\mu q_t)$ \\
				%
				$\mathcal{O}_{qq}^{(3)}$  & $(\bar q_p \gamma_\mu \tau^I q_r)(\bar q_s \gamma^\mu \tau^I q_t)$ \\
				%
				$\mathcal{O}_{lq}^{(1)}$                & $(\bar l_p \gamma_\mu l_r)(\bar q_s \gamma^\mu q_t)$ \\
				%
				$\mathcal{O}_{lq}^{(3)}$                & $(\bar l_p \gamma_\mu \tau^I l_r)(\bar q_s \gamma^\mu \tau^I q_t)$ 
			\end{tabular}
		\end{minipage}
		%
		%
		\begin{minipage}[t]{5.45cm}
			\renewcommand{\arraystretch}{1.5}
			\begin{tabular}[t]{c|c}
				\multicolumn{2}{c}{$(\bar RR)(\bar RR)$} \\
				\toplinetwo
				$\mathcal{O}_{ee}$               & $(\bar e_p \gamma_\mu e_r)(\bar e_s \gamma^\mu e_t)$ \\
				%
				$\mathcal{O}_{uu}$        & $(\bar u_p \gamma_\mu u_r)(\bar u_s \gamma^\mu u_t)$ \\
				%
				$\mathcal{O}_{dd}$        & $(\bar d_p \gamma_\mu d_r)(\bar d_s \gamma^\mu d_t)$ \\
				%
				$\mathcal{O}_{eu}$                      & $(\bar e_p \gamma_\mu e_r)(\bar u_s \gamma^\mu u_t)$ \\
				%
				$\mathcal{O}_{ed}$                      & $(\bar e_p \gamma_\mu e_r)(\bar d_s\gamma^\mu d_t)$ \\
				%
				$\mathcal{O}_{ud}^{(1)}$                & $(\bar u_p \gamma_\mu u_r)(\bar d_s \gamma^\mu d_t)$ \\
				%
				$\mathcal{O}_{ud}^{(8)}$                & $(\bar u_p \gamma_\mu T^A u_r)(\bar d_s \gamma^\mu T^A d_t)$ \\
				%
			\end{tabular}
		\end{minipage}

		\begin{minipage}[t]{4.95cm}
			\renewcommand{\arraystretch}{1.5}
			\begin{tabular}[t]{c|c}
				\multicolumn{2}{c}{$(\bar LL)(\bar RR)$} \\
				\toplinetwo
				$\mathcal{O}_{le}$               & $(\bar l_p \gamma_\mu l_r)(\bar e_s \gamma^\mu e_t)$ \\
				%
				$\mathcal{O}_{lu}$               & $(\bar l_p \gamma_\mu l_r)(\bar u_s \gamma^\mu u_t)$ \\
				%
				$\mathcal{O}_{ld}$               & $(\bar l_p \gamma_\mu l_r)(\bar d_s \gamma^\mu d_t)$ \\
				%
				$\mathcal{O}_{qe}$               & $(\bar q_p \gamma_\mu q_r)(\bar e_s \gamma^\mu e_t)$ \\
				%
				$\mathcal{O}_{qu}^{(1)}$         & $(\bar q_p \gamma_\mu q_r)(\bar u_s \gamma^\mu u_t)$ \\ 
				%
				$\mathcal{O}_{qu}^{(8)}$         & $(\bar q_p \gamma_\mu T^A q_r)(\bar u_s \gamma^\mu T^A u_t)$ \\ 
				%
				$\mathcal{O}_{qd}^{(1)}$ & $(\bar q_p \gamma_\mu q_r)(\bar d_s \gamma^\mu d_t)$ \\
				%
				$\mathcal{O}_{qd}^{(8)}$ & $(\bar q_p \gamma_\mu T^A q_r)(\bar d_s \gamma^\mu T^A d_t)$\\
			\end{tabular}
		\end{minipage}
		%
		%
		\begin{minipage}[t]{5.45 cm}
			\renewcommand{\arraystretch}{1.5}
			\begin{tabular}[t]{c|c}
				\multicolumn{2}{c}{$(\bar LR)(\bar L R)+\hbox{h.c.}$} \\
				\toplinetwo
				$\mathcal{O}_{quqd}^{(1)}$ & $(\bar q_p^j u_r) \epsilon_{jk} (\bar q_s^k d_t)$ \\
				%
				$\mathcal{O}_{quqd}^{(8)}$ & $(\bar q_p^j T^A u_r) \epsilon_{jk} (\bar q_s^k T^A d_t)$ \\
				%
				$\mathcal{O}_{lequ}^{(1)}$ & $(\bar l_p^j e_r) \epsilon_{jk} (\bar q_s^k u_t)$ \\
				%
				$\mathcal{O}_{lequ}^{(3)}$ & $(\bar l_p^j \sigma_{\mu\nu} e_r) \epsilon_{jk} (\bar q_s^k \sigma^{\mu\nu} u_t)$\\
				%
			 $\mathcal{O}_{ledq}$ & $(\bar l_p^j e_r)(\bar d_s q_{tj})$ 
			\end{tabular}
		\end{minipage}
	\end{center}
		\vspace{-.35cm}
	\caption{\label{warsaw}
	  Complete list of the dimension-six SMEFT operators in the Warsaw basis 
		\cite{Grzadkowski:2010es}. The $\mathcal{CP}$ violating operators contains the dual fields~$\widetilde X$. The flavour labels of the form $p,r,s,t$ on the $\mathcal{O}$ operators are suppressed on the left hand side of
		the tables.}
\end{table}
%%%%%%%%%%%
\par Phenomenological studies of EFTs with dimension-six operators primarily focus on using a set of complete and non-redundant ``basis''. This is since different effective operators will correspond to the same observables, e.g. same scattering amplitudes of SM particles.  This is the case if the operators can be related using equations of motion, Fierz transformations, integration by parts or field redefinitions. Thus leading to non-trivial and counter-intuitive relations between operators. Consequently, the construction of basis for the dimension-six SMEFT Lagrangian of eq.~\eqref{smeftdim6} is a cumbersome task. Such task has been accomplished by~\cite{Grzadkowski:2010es} recently forming what is known as the \textbf{Warsaw Basis}.  Another set of basis is the strongly-interacting light Higgs basis (SILH), initially proposed by~\cite{Giudice:2007fh}, before the Warsaw basis and completed in refs.~\cite{Contino:2013kra, Elias-Miro:2013eta}. A more recent set of basis has been published in~\cite{Gupta:2014rxa} using a subset of couplings characterising the interactions of mass eigenstates in the effective Lagrangian.
\par The complete $d=6$ SMEFT is described by 2499 independent parameters~\cite{Jenkins:2013zja,Jenkins:2013wua,Alonso:2013hga}. However, if one suppresses the flavour indices, assuming SMEFT is flavour universal, their inventory is significantly reduced. In the Warsaw basis, for example, assuming Baryon number conservation and dropping the flavour indices, one has only 59 operators, listed in~\autoref{warsaw}. It should be noted that all of the SMEFT basis will produce the same phenomenology, though the choice of basis is sometimes helpful in simplifying the analysis. In this thesis, I will focus on Warsaw basis.\\ 
\subsection{Single Higgs processes in SMEFT}
Single Higgs production and decay processes are modified at LO by a relatively long list of operators summarised in~eqs.~\eqref{box:smefthiggslo}, ~\eqref{box:smefthiggslo2} and~\eqref{box:smefthiggslo3}. Explicit formulae for the Higgs rates dependence on the Wilson coefficients of these operators can be found in~\cite{ATLAS:2019dhi}
\begin{tcolorbox}[title=SMEFT operators modifying Higgs rates at LO,
	title filled=false,
	colback=Mahogany!5!white,
	colframe=Mahogany ]
	Higgs operators
	\begin{align}
		C_{\phi D}, \ \mathcal{O}_{\phi \Box},\ \mathcal{O}_{\phi G}, \ \mathcal{O}_{\phi W},\ \mathcal{O}_{\phi B}, \ \mathcal{O}_{\phi W B},\ \mathcal{O}_{\phi l}^{(1)}, \nn \\
		\ \mathcal{O}_{\phi l}^{(3)}, \ \mathcal{O}_{\phi e},\ \mathcal{O}_{\phi q}^{(1)},\ \mathcal{O}_{\phi q}^{(3)}, \  \mathcal{O}_{\phi u}, \ \mathcal{O}_{\phi d},\ \mathcal{O}_{\tau \phi}, \ \mathcal{O}_{t \phi}, \ \mathcal{O}_{b \phi},\ \mathcal{O}_{tb \phi}.
		\label{box:smefthiggslo}
	\end{align}
	Top-quark operators
	\begin{equation}
		\mathcal{O}_{t G}, \ \mathcal{O}_{t W}, \ \mathcal{O}_{t B},
		\label{box:smefthiggslo2}
	\end{equation}
	other 
	\begin{equation}
		\mathcal{O}_G,\ \mathcal{O}_{ll}^{(1)},\ \mathcal{O}_{Qq}^{(1),(3)},\ \mathcal{O}_{tu},\ \mathcal{O}_{td}^{(1),(8)},\ \mathcal{O}_{Qu}^{(1),(8)}, \ \mathcal{O}_{Qd}^{(1),(8)}.
		\label{box:smefthiggslo3}
	\end{equation}
	The third-generation quarks are denoted by~$Q$ while the first and second-generation quarks are assumed to have the same coupling and are denoted by $q,u,d$.
\end{tcolorbox}
Some of these operators are strongly constrained from EWPO data such as~$\mathcal{O}_{\phi D}$ and $ \mathcal{O}_{\phi W B}$, while others still have weak bounds from current measurements and do not affect EWPOs. A most recent fit on SMEFT Wilson coefficients can be found in ref.~\cite{Dawson:2020oco}, where Higgs and EW data were used to fit a subset of the SMEFT Wilson coefficients of the operators listed above. The fit also includes the effects of RGE and NLO (even NNLO corrections to $m_W$). Instead, in~\cite{Ethier:2021bye}, a global fit for a larger set of operators, but only including LO effects, including EW, Higgs and top-quark data.  A study that was published in ref.~\cite{Dawson:2022bxd}, has utilised EWPO data to constrain the four-fermion operators appearing in Higgs rates at LO and operators with four heavy quarks, using their NLO effects on EW bosons pole masses. We shall see in~\autoref{chap:4topSingleHiggs} that the latter operators also contribute to Higgs rates at NLO. A wider scope analysis including a wide range of Higgs, top-quark, di-boson and EWPO data has been performed in~\cite{Ellis:2020unq}. 
\par The dependence of single Higgs rates on the SMEFT Wilson coefficients gets more complicated once higher-order effects are taken into account. In the fit results reported from~\cite{Dawson:2020oco}, the RGE of these Wilson coefficients introduces mixing with operators that do not appear at LO, and also the non-log piece of the loop corrections to the rates and masses of the EW and Higgs bosons, see for example refs.\cite{Passarino:2012cb,Degrassi:2016wml,Kribs:2017znd}. \\ A prominent example of an operator appearing only at NLO in single Higgs processes is $\mathcal O_\phi$, which modifies the Higgs self-interactions, namely the trilinear coupling. 
Typically, one needs to observe Higgs pair production to directly probe the Higgs trilinear self-coupling. However, due to the appearance of Higgs self-interaction and its modifiers, i.e.~$C_\phi$ in SMEFT context, in higher-order EW corrections to Higgs observables and EWPO data~ \cite{McCullough:2013rea, Gorbahn:2016uoy, Degrassi:2016wml, Bizon:2016wgr ,Kribs:2017znd,Maltoni:2017ims, Degrassi:2019yix, Degrassi:2021uik, Haisch:2021hvy}. \autoref{fig:h_nlo_ew} illustrates example Feynman diagrams of single Higgs processes to which the trilinear Higgs self-coupling enters via NLO corrections.
\begin{figure}[htpb!]
	\begin{center}
		\includegraphics[width=0.8\textwidth]{figures/htoaa_nlo_ew}
		\caption{NLO EW corrections of single Higgs processes,  were the Higgs trilinear self-coupling~(the red circle) enters. Here the Higgs decay to two photons is shown as an example. \label{fig:h_nlo_ew} }
	\end{center}
\end{figure}
Using the results from the aforementioned references, a global fit with all operators that enter at tree-level in addition to the loop effects from the Higgs self-coupling has been performed in refs.~\cite{DiVita:2017eyz,Dawson:2020oco}. Additionally, experimental searches for the Higgs trilinear self-coupling in single-Higgs rates have been presented by ATLAS~\cite{ATLAS:2019pbo} and CMS \cite{CMS:2020gsy}.\footnote{I present references here to the most recent results.} 
%%%%%%%
\subsection{Higgs pair production and SMEFT}
\par Higgs pair production in hadron colliders is sensitive to six~$\mathcal{CP}$ even SMEFT operators \footnote{For or Higgs pair production with~$\mathcal{CP}$ violating operators, see ref.~\cite{Grober:2017gut}. }, under the assumption of Minimal Flavour violation~(MFV).~\footnote{MFV assumes that new physics operators will follow the same flavour hierarchies as the SM.} These operators are
\begin{equation}
	\mathcal{O}_{\phi D},\ \mathcal{O}_{\phi \Box},\ \textcolor[HTML]{4334a0}{\mathcal{O}_{\phi}}, \ 	\textcolor[HTML]{2ebbaa}{\mathcal{O}_{t\phi}}, \ 	\textcolor[HTML]{ae0034}{\mathcal{O}_{\phi G}},\ \textcolor[HTML]{ffb743}{\mathcal{O}_{t G}},
	\label{HH-smeft}
\end{equation}
and their effects, with the corresponding colours are demonstrated in~\autoref{fig:hh-smeftw}, except for~$\mathcal{O}_{\phi D}$ and  $\mathcal{O}_{\phi \Box}$, as they modify all SM Higgs vertices. 
\begin{figure}[t!]
	\begin{center}
		\includegraphics[width=0.75\textwidth]{figures/hh-smeft}
		\caption{ Example of diagrams illustrating how the dimension-six SMEFT operators enter in Higgs pair production at hadron colliders. \label{fig:hh-smeftw} }
	\end{center}
\end{figure}
However, MFV is not the only way to approach SMEFT, there exist more complex flavour structures that allow for significant enhancements of the first and second generation Yukawa couplings without being excluded by flavour observables. Such formalisms will be discussed in~\autoref{chap:lightyuk}.
\par The primary operator to constrain from Higgs pair as mentioned before is $\mathcal{O}_{\phi}$, for two reasons; a) the rest of the operators appearing in di-Higgs can be strongly constrained from single Higgs and top quark processes. b) The effect of $\mathcal{O}_{\phi}$ on Higgs pair production is significantly higher than in single Higgs or EW observables. This is illustrated in~\autoref{fig:hh-vs-h} by comparing the relative change of the gluon fusion cross-sections at NLO QCD for single and di-Higgs production. This is not surprising since $C_\phi$ appears at LO in Higgs pair production.
\begin{figure}[h!]
	\begin{center}
		\includegraphics[width=0.65\textwidth]{figures/trilinear_single_vs_double}
		\caption{ The relative change of the NLO QCD cross-section of gluon fusion production of single Higgs (dashed line) and Higgs pair (solid line) at a $pp$ collider with $\sqrt{s}=14$ TeV as a function of $C_\phi$ or the corresponding $\kappa_\lambda$. \label{fig:hh-vs-h} }
	\end{center}
\end{figure}
\par Another advantage for Higgs pair production searches is the sensitivity of this process to non-linear couplings, for example, diagrams (b) and (d) of ~\autoref{fig:hh-smeftw}. Although in SMEFT, these diagrams correspond to the same operators in (a) and (c), respectively, in HEFT, this is not necessarily the case.
\section{The Higgs effective field theory\label{sec:chiral}}
Given the strong bonds on the $\rho$ parameter, it would be plausible to assume that the NP maintains the custodial symmetry~$SU(2)_V$ and treats the chiral symmetry breaking pattern $SU(2)_L \otimes SU(2)_R \to SU(2)_V$  the same way the QCD chiral symmetry breaking is treated. This formalism considers the pions as pseudo-Nambu Goldstone bosons to describe their properties and couplings. In the pion case, this is known as \textbf{chiral perturbation theory}~\cite{GASSER1984142, GASSER1985465}. The same mathematical description could be applied to the case of EW symmetry breaking by constructing the EW chiral Lagrangian~(EWChL).   In this formalism, the Goldstone bosons~$\pi^a(x)$ of the SM are considered the generators of $SU(2)_L$ unitary transformation.
\begin{equation}
	\mathcal U(x) = e^{ i \pi^a(x)\sigma_a/v }, 
\end{equation}
which implies that the Goldstone fields transform non-linearly under~$SU(2)_L \otimes SU(2)_R$.  The Higgs boson~$h(x)$ is added as an $SU(2)_L \otimes U(1)_Y$ singlet, and can appear in the EWChL at any power. Contrary to the SMEFT power counting in the NP scale $ \Lambda$, in the EWChL, terms are ordered according to their \emph{chiral dimension} $\chi$, defined for spacetime derivatives $\partial_\mu$,  bosonic $\phi, X_\mu$ and $\psi$ fermionic generic fields as~\cite{Buchalla:2013rka,Buchalla:2015wfa}
\begin{equation}
	[\phi]_\chi =0,\,\, [X]_\chi =0,\,\, [\partial_\mu ]_\chi =1, \,\, [\psi]_\chi =2.
\end{equation}
 \par The zeroth-order term of the EWChL possesses a chiral dimension of $\chi=2$, while higher-order terms could be considered terms generated perturbatively from $L$ loop interactions, an having a chiral dimension~$\chi= 2L+2$.This power-counting causes some SMEFT dimension-six operators, in the Warsaw basis, to be considered of a higher order in EWChL. A prominent example of this is the chromomagnetic operator~$\mathcal O_{tG}$ being of chiral dimension five. 
The relevant terms for single- and di-Higgs production of the EWChL are given in the unitary gauge by~\cite{LHCHiggsCrossSectionWorkingGroup:2016ypw,DiVita:2017eyz}
\begin{align}\nn
	\mathcal{L}_{\mathrm{HEFT}} = & \, \frac{h}{v} \Bigg[  \left( \delta c_W m_W^2 W_{\mu}^+W^{-\mu} +\delta c_Z \frac{m_Z^2}{2} Z_\mu Z^\mu\right)  \\\nn
	&+c_{ww}\frac{g_2^2}{2}W_{\mu\nu}^+W^{-\mu\nu} + c_{w\square} g_2^2\left(W_{\mu}^-\partial_\nu W^{+\mu\nu} + \text{h.c.}\right) +  c_{\gamma\gamma}\frac{\alpha}{8 \pi}A_{\mu\nu}A^{\mu\nu} \\\nn
	& +c_{zz}\frac{g_2^2 + g_1^2}{4} Z_{\mu\nu}Z^{\mu\nu}+ c_{z\gamma}\frac{e g_1}{16 \pi^2}Z_{\mu\nu}A^{\mu\nu}
	+c_{z\square}g_2^2Z_\mu\partial_\nu Z^{\mu\nu}
	+c_{\gamma \square}g_2 g_1 Z_\mu\partial_\nu A^{\mu\nu}
	\Bigg]\\\nn
	&+ \frac{\alpha_s}{8 \pi} \left( c_{gg} \frac{h}{v} +  c_{gg}^{(2)} \frac{h^2}{2 v^2}\right) \Tr[G_{\mu\nu} G^{\mu\nu}]
	-\sum_f \left[ m_f \left(c_f \frac{h}{v} + c _{ff} \frac{h^2}{2v^2}\right) \bar{f}_{R}f_{L}+\text{h.c.}\right]\\
	& - c_{hhh} \frac{m_h^2}{2 v} h^3\ + \dots\label{eq:coupl_def}
\end{align}
I have omitted here the kinetic and mass terms of the Higgs, $\mathcal{CP}$ violating terms, as well as couplings not relearnt to LHC phenomenology. \\
In addition to NP effects, this Lagrangian also includes the LO and NLO SM vertices, for example the parameter $\delta c_V=1$ corresponds to the tree-level coupling between the Higgs field and the EW bosons~$ V=W, Z$. While the coupling $c_{gg}= 2/3$ corresponds to the SM effective coupling at NLO if the heavy top limit~(HTL)~$m_t \to \infty$ is applied. 
In contrast to the SMEFT, the couplings of one and two Higgs bosons to fermions or gluons become de-correlated. This feature gives this Lagrangian a richer phenomenology for Higgs pair production.  
\par The HEFT coefficients modifying the Higgs pair production via gluon fusion are 
\begin{equation}
	\textcolor[HTML]{4334a0}{c_{hhh} }, \ 	\textcolor[HTML]{2ebbaa}{c_t}\ (a) , \  	\textcolor[HTML]{2ebbaa}{c_{tt}} \ (b), \  \textcolor[HTML]{ae0034}{c_{gg}} \ (c), \  \textcolor[HTML]{ae0034}{c_{gg}^{(2)}} \ (d),
\end{equation}
with the same colours highlighted in the operator insertions of~\autoref{fig:hh-smeftw} and where the letter next to the coefficient indicates the diagram, in which the coefficient appears.  A full parametrisation of the Higgs pair cross-section at NLO (inclusive and differential) and NNLO (inclusive) can be found in refs.~\cite{Buchalla:2018yce,Capozi:2019xsi,deFlorian:2021azd} and is implemented at NLO in the \texttt{POWHEG-BOX} software~\cite{Heinrich:2020ckp}. \\
% UV-complete models that are related to the EWChL are composite Higgs models~\cite{Contino:2010rs,Panico:2015jxa,AGASHE2005165}, dilaton theories~\cite{PhysRevLett.100.111802}, techni-dilaton models~\cite{Habaa:2010rbs}, technicolour models~\cite{Delgado:2010bb} and other models with induced EW symmetry breaking~\cite{Galloway:2013dma,Chang:2014ida}.
\subsection{Translation between SMEFT and HEFT }
\par In order to have a canonical Higgs boson propagator and facilitate the translation between SMEFT and HEFT or to the $\kappa$-formalism, one needs to put the SMEFT Lagrangian into the canonical form, that is to convert the operators with covariant derivatives acting on the Higgs to canonically normalised Higgs kinetic term. This is done done by the field redefinition.
\begin{equation}
	\phi=\left( \begin{array}{c} 0 \\ h(1+c_{h,kin}) + v \end{array} \right),
\end{equation} 
with 
\begin{equation}
	c_{h,kin}=\left(C_{\phi,\Box}-\frac{1}{4}C_{\phi D}\right) \frac{v^2}{\Lambda^2}\,.
\end{equation}
This field redefinition will generate derivative interactions of the form $h(\partial_{\mu}h)^2$ and $h^2(\partial_{\mu}h)^2$. In order to remove these terms, and for sake of simplicity, I use a gauge-dependent field redefinition\footnote{For gauge-independent formalism cf.~\cite{Hartmann:2015aia}.}
\begin{equation}
	h \to h + c_{h,kin}\left( h +\frac{h^2}{v}+\frac{h^3}{3v^2}\right)\,. \label{fieldref}
\end{equation}
This field redefinition leads to n $c_{h,kin}$ modifying all Higgs couplings. 
\par Before discussing the translation between SMEFT and HEFT, some words of caution are in order: First, HEFT is less restrictive than SMEFT. Therefore, it contains more degrees of freedom. This makes some points of the HEFT parameter space unmappable to SMEFT. In addition, the operator ordering is different in both formalisms, as mentioned before. Some operators present in SMEFT will be absent in HEFT and vice-versa.  In ~\autoref{tab:translation}, the translation between the HEFT and SMEFT Wilson coefficients of the operators relevant to Higgs pair production at LO is shown. 
\begin{table}[htb]
	\begin{center}
		\begin{tabular}{ c c }
			\toplinetwo
			HEFT& SMEFT (Warsaw)\\
			\midrule
			$c_{hhh}$&$1-2\frac{v^4}{m_h^2}C_\phi+3c_{h,kin}$ \\
			$c_f$ & $1+c_{h,kin} -C_{f\phi} \frac{v^3}{\sqrt{2} m_f}$\\
			$ c_{ff} $ &$-C_{f\phi} \frac{3 v^3}{2\sqrt{2} m_f} + c_{h,kin}$\\
			$c_{gg}$  & $8\pi/\alpha_s v^2 C_{\phi G}$ \\
			$c_{gg}^{(2)}$  & $4\pi/\alpha_s v^2 C_{\phi G}$ \\
			\bottomrule
		\end{tabular}
	\end{center}
	\caption{Translation between the Wilson coefficients of HEFT and SMEFT for the operators relevant to Higgs pair production. \label{tab:translation}}
\end{table}
More general translation between SMEFT in Warsaw and SILH basis and HEFT can be done automatically using \texttt{Rosetta} package~\cite{Falkowski:2015wza}
\subsection{EFT and $\kappa$-formalism \label{eftkappa}}
\par The $\kappa$-formalism provides an experimentally accessible approach to study the Higgs boson properties. The $\kappa$ parameters are part of a more generalised formalism called the Higgs \textbf{Pseudo-observables}~\cite{Gonzalez-Alonso:2014eva}. 
If the new physics contributions do not generate new Lorentz structures, there is a possible translation between the Wilson coefficients in the SMEFT Warsaw basis and the $\kappa$ formalism. In particular, taking the rescaling of the trilinear coupling, $\kappa_\lambda$, the translation is given by
\begin{equation}
	\kappa_\lambda = 1-\frac{2v^4}{m_h^2} \frac{C_\phi}{\Lambda^2}+3 c_{h,kin}.
\end{equation}
A similar relation exists for the rescaling of the quark Yukawa couplings~$\kappa_q$
\begin{equation}
	\kappa_q = 1+c_{h,kin}- \frac{v^3}{\sqrt{2}m_q}\frac{C_{q\phi}}{\Lambda^2}.
\end{equation}
\par In these two examples, one can see the similarities between $\kappa$-formalism and HEFT, but this is not always the case.  Other translations could be obtained by comparing how SMEFT operators modify the Higgs couplings with the SM and matching it with the corresponding $\kappa$ or other Higgs pseudo-observable.\\ 
However, one should be careful while interpreting results quoted in terms of Wilson coefficients in the SMEFT framework extracted from multi-Higgs or multi-vector bosons searches. These results include couplings that are not present in the SM. For example, the $hh q\bar{q}$ coupling, though being linearly related to the quark Yukawa coupling $h q\bar{q}$, is not a rescaling of any SM Higgs coupling. With this in mind, one can strictly remain within a linear EFT and link the rescaling of the quark Yukawa, $\kappa_q$, to the~$hh q\bar{q}$ coupling through
\begin{equation}
	g_{hhq\bar{q}}^{\mathrm{linear-EFT}} = -\frac{3}{2}\frac{1-\kappa_q}{v} \, g_{h q\bar{q}}^{\mathrm{SM}}.
\end{equation}
This relation will no longer hold once a non-linear EFT, like HEFT, is used. Hence, the $\kappa$-formalism must be applied carefully when multi-Higgs signals are considered.
\section{Conclusions \label{sec:concefts}}
\par Effective field theories provide a systematic yet simplified approach for NP searches by simplifying its complex interaction structures. This can be viewed as a dimensionality reduction approach and collapsing all the NP interactions into effective ones. They would be observed at colliders with energy reaches below the NP scale $\Lambda$. The linear approach to EFT is called the SMEFT that preserves the SM fields and symmetries, and the Higgs boson is a part of an $SU(2)_L$ doublet $\phi$ like the SM. In contrast, non-linear approaches such as HEFT/EWChL treat the Higgs boson as a singlet. The latter approach is more general and introduces independent parameters involving multiple Higgs bosons. For example, the couplings $f\bar f h$ and $ f\bar f hh$ will be generated in SMEFT and HEFT. Still, in SMEFT, both are related by the Wilson coefficient $C_{\phi f}$ \footnote{They are also related by the coefficient $c_{h,kin}$ that modifies all couplings of the Higgs boson}, while in HEFT, they have independent  Wilson coefficients $c_f$ and $c_{ff}$, respectively. 
\par Most of the Wilson coefficients involving Higgs interactions are strongly constrained by EWPOs and Higgs and top-quark data. However, the bounds on the Wilson coefficient modifying Higgs self-couplings $C_\phi$ remain dominated by theoretical constraints from perturbative unitarity~\cite{DiLuzio:2017tfn,DiVita:2017vrr}. This can be improved by the searches for Higgs pair production at the HL-LHC, as this process is more sensitive to the trilinear Higgs self-coupling than EWPO and single-Higgs data.
%\begin{figure}[htbp!]
%	\begin{center}
	%		\includegraphics[width=\textwidth]{figures/smeft_heatmap}
	%		\caption{ \label{fig:greatheatmap} }
	%	\end{center}
%\end{figure}
%In~\autoref{greatheatmap}, I show the best bounds on the Wilson coefficients relearnt to Higgs production as well as heavy quark four-fermion operators, with a heatmap indicating the contribution of each operator in prominent Higgs, top and EW precision observables.  Although this is a subset of the total SMEFT operators and observables used in the fits, one can see the interconnectivity of the measurements.\\ The main objective of this thesis is to extend these connections by exploiting the potential of single-Higgs data and Higgs pair production to constrain the Higgs trilinear coupling modifiers (mainly in SMEFT) and the interplay between $C_\phi$ and heavy quark four-fermion operators in single Higgs data. Moreover, the SMEFT picture can be further extended by unravelling the interplay between Light quark couplings modifiers in Higgs pair production. Lastly, I will show another connection between Higgs operators in SMEFT and flavour anomalies.  Emphasising the complex interconnectivity between experimental observables and SMEFT operators. 
%%%%%%%

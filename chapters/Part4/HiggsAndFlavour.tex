%!TEX encoding = UTF-8 Unicode
% !TeX spellcheck = en_GB

%%%%%%%%%%%%%%%%%%%%%%%%%%%%%%%%%%%%%%
\chapter{ Higgs and flavour }\label{chap:flavhiggs}
%%%%%%%%%%%%%%%%%%%%%%%%%%%%%%%%%%%%%%
%%%%%%%%%%%%%%%%%%%%%%%%%%%
\section{Effective Field Theory for Higgs pair production}
\label{sec:flavEFT}
%%%%%%%%%%%%%%%%%%%%%%%%%%%

The potential deformations of the SM in a model-independent manner can be accomplished by means of an EFT description parametrising new physics (NP) with higher-dimensional operators suppressed by some large energy scale $\Lambda$. A complete basis for the higher-dimensional operators has been given in Refs.~\cite{Grzadkowski:2010es,Contino:2013kra}. In this work we are interested in probing the Higgs trilinear and light-quark Yukawa couplings. 
Starting with the dimension-six operators modifying the Higgs self-couplings, we see that they are given by 
\begin{align}
	\mathcal{L} \supset &
	\frac{C_{\phi\Box}}{\Lambda^2}\,(\phi^\dagger \phi)\Box(\phi^\dagger \phi)+\frac{C_{\phi D}}{\Lambda^2}\,(\phi^\dagger D_\mu \phi)^*(\phi^\dagger D^\mu \phi)+\frac{C_\phi}{\Lambda^2}|\phi^{\dagger} \phi|^3.
	\label{eq:EFTop}
\end{align}
where $\phi$ denotes the Higgs-doublet which, in the unitary gauge, can be written as $\phi=1/\sqrt{2}(0,v+h)^T$. 
It is common to quote the constraints on the Higgs couplings in terms of rescaling to the SM coupling prediction, typically denoted by $\kappa$:
\begin{equation}
	\kappa = \frac{g_h}{g_h^{\mathrm{SM}}}\,.
\end{equation}
If the new physics contributions do not generate new Lorentz structures there is a possible translation between the Wilson coefficients in the SMEFT Warsaw basis discussed above, and the $\kappa$ formalism usually used by experimentalists. In particular, taking the rescaling of the trilinear coupling, $\kappa_\lambda$, the translation is given by
\begin{equation}
	\kappa_\lambda = 1-\frac{v^4}{m_h^2} \frac{C_\phi}{\Lambda^2}+3 c_{\phi,\mathrm{kin}},
\end{equation}
where $c_{\phi,\mathrm{kin}}$ is given by
\begin{equation}
	c_{\phi,\mathrm{kin}} = \left( C_{\phi\Box} -\frac{1}{4} C_{\phi D}\right) \frac{v^2}{\Lambda^2}.
\end{equation}
The latter Wilson coefficients modify all the Higgs couplings, and are strongly constrained by electroweak precision observables (e.g.~the $T$ parameter constrains $C_{\phi D}$).Therefore, we set~$c_{\phi,\mathrm{kin}}=0$ from now on.\\
%%%%%%%%%%
Before discussing the SMEFT operators modifying the coupling between the Higgs boson and light quarks, we start by a review of the SM couplings between them, i.e. the Yukawa interaction
\begin{equation}
	-\mathcal{L}_{y}=y^u_{ij} \bar{q}_L^i \tilde{\phi} u_R^j + y^d_{ij} \bar{q}_L^i \phi d_R^j +h.c.\,,
	\label{eq:yukawa}
\end{equation}
Here, $q_L^i$ is the left-handed $SU(2)$ quark doublet of the $i^{th}$ generation and $u_R^j$ and $d_R^j$, the right-handed up- and down-type fields of the $j^{th}$ generation, respectively, and $\tilde{\phi}=i \sigma_2 \phi^*$. \\ The $3 \times3$ Yukawa matrices are the SM \textbf{spurions} that break the flavour symmetry of the SM~$U(3)_{Q_L}\otimes U(3)_{u_R}\otimes U(3)_{d_R}$ to the baryon number and the gauged hypercharge groups, i.e. $U(1)_B\otimes U(1)_Y$. In the ground state, the the Lagrangian~\eqref{eq:yukawa} gives  the quark masses. Thus, defining a diagonal \emph{mass basis} as opposed to a generic~\emph{interaction basis} that eq~~\eqref{eq:yukawa} is written in it. 
The transformation of the yukawa matricies~$y^{u/d}$ from generic flavour basis to the mass basis~$Y^{u/d} = \mathrm{diag}(y_1^{u/d},y_2^{u/d},y_3^{u/d})$ is performed by means of a bi-unitary transformation. To illustrate this we show the singular-value decomposition of the Yukawa matrices
\begin{align}
	(y^u)_{ij}&= (\mathcal{U}_{L}^{u})_{li} (Y^{u})_{ll} (\mathcal{U}_R^{u})^\dagger_{lj},\nonumber \\
	(y^d)_{ij}&=(\mathcal{U}_{R}^{d})^\dagger_{li} (Y^{d})_{ll} (\mathcal{U}_L^{d})_{lj}.
	\label{eq:yuksvd}
\end{align}
This decomposition is not unique and only defined upto a $U(1)^5$ and a $U(1)_B$ phases. However, this transformation freedom does not hold for CKM matrix, defined as
\begin{equation}
	V_{CKM} = (\mathcal{U}_{L}^{u})^T (\mathcal{U}_L^{d})^*,
\end{equation}
where we can only rotate by $U(1)_B$. This manifests in Flavour-changing charged currents at tree-level. But no Flavour-changing neutral currents~(FCNC) are allowed at tree-level in the SM. Additionally, the loop-induced FCNC's are CKM suppressed\\
%%%%
In a similar manner, the SMEFT introduces new flavour spurions via the dimension-six operators
\begin{align}
	\mathcal{L} \supset \frac{\phi^{\dagger}\phi}{\Lambda^2}\left( (C_{u\phi})_{ij} \bar{q}_L^i \tilde{\phi} u_R^j + (C_{d\phi})_{ij} \bar{q}_L^i \phi d_R^j +h.c.\right)\,,
\end{align}
%%%%%%
The mass matrices of the up- and down-type quarks are obtained from the Yukawa and the new SMEFT coupling 
\begin{align}
	M^u_{ij} =& \frac{v}{\sqrt{2}} \left( y^u_{ij}-\frac{1}{2} (C_{u\phi})_{ij}\frac{v^2}{\Lambda^2}\right)\,,\nonumber\\
	M^d_{ij} =& \frac{v}{\sqrt{2}} \left( y^d_{ij}-\frac{1}{2} (C_{d\phi})_{ij}\frac{v^2}{\Lambda^2}\right)\,. \label{eq:mass}
\end{align}
%%%%%%%% 
The Wilson coefficients' matrices~$C_{q\phi}$ need not to be simultaneously diagonalisable with the SM Yukawa's~$y^q$. However, we need to have a diagonal mass basis like the SM ones, here we will be having new set of bi-unitary transformations~$\mathcal{V}_{L/R}^{u/d}$ such that we could write~$C_{q\phi}$ in terms of the mass basis ones~$\tilde{C}_{q\phi}$ in a similar ways to   eq~\eqref{eq:yuksvd}
\begin{align}
	(C_{u\phi})_{ij}&= (\mathcal{V}_{L}^{u})_{li}(\tilde{C}_{u\phi})_{lm} (\mathcal{V}_R^{u})^\dagger_{mj},\nonumber \\
	(C_{d\phi})_{ij}&=(\mathcal{V}_{R}^{d})^\dagger_{li} (\tilde{C}_{d\phi})_{lm} (\mathcal{V}_L^{d})_{mj}.
	\label{eq_defV}
\end{align}
where~$\mathcal{V}_{L/R}^{u/d}$ are only guaranteed to diagonalise the mass matrices~$M^{u/d}_{ij}$ in general~\footnote{The CKM matrix with this extended flavour sector will not longer guaranteed to be unitary, however unitarity violation will be typically of order $m_q^2/\Lambda^2$.  }.   \\
The couplings of one and two Higgs boson to fermions can be defined as~(in the mass basis)
\begin{equation}
	\mathcal{L}\supset g_{h\bar{q}_i q_j}\bar{q}_i q_j h + g_{h\bar{q}_i q_j}\bar{q}_i q_j h^2\,,
\end{equation} 
with
\begin{equation}
	g_{h\bar{q}_i q_j} := \quad \frac{m_{q_i}}{v}\delta_{ij}-\frac{v^2}{\Lambda^2} \frac{(\tilde{C}_{q\phi})_{ij}}{\sqrt{2}}\,, \quad \quad \quad \quad \quad g_{h h\bar{q}_i q_j} := \quad -\frac{3}{2\sqrt{2}}\frac{v}{\Lambda^2}(\tilde{C}_{q\phi})_{ij}\,. 
	\label{eq:couplingsEFT}
\end{equation}
A similar relation exists for the rescalings of the quark Yukawa couplings~$\kappa_q$
\begin{equation}
	\kappa_q = 1- \frac{v^3}{\sqrt{2}m_q}\frac{C_{q\phi}}{\Lambda^2}.
\end{equation}
However, one should be careful while interpreting results quoted in terms of Wilson coefficients in the SMEFT framework extracted from di-Higgs, multi-Higgs or multi-vector bosons searches, as these results include couplings that are not present in the SM. For example, the $hh q\bar{q}$ coupling, though being linearly related to the quark Yukawa coupling $h q\bar{q}$, is not a rescaling of any SM Higgs coupling.  With this in mind, one can strictly remain within a linear EFT and link the rescaling of the quark Yukawa, $\kappa_q$, to the~$hh q\bar{q}$ coupling through
\begin{equation}
	g_{hhq\bar{q}}^{\mathrm{linear-EFT}} = -\frac{3}{2}\frac{1-\kappa_q}{v} \, g_{h q\bar{q}}^{\mathrm{SM}}.
\end{equation}
This relation will no longer hold once a non-linear EFT is used. Hence, the $\kappa$-formalism, in a strict sense, is not  applicable to multi-Higgs studies.\\
Here, we see that the new Wilson coefficients introduce tree-level FCNC, or even if a tree-level ones are suppressed the loop-levels ones will not have the SM CKM suppression.\\
Generically, such a construction leads to flavour-changing neutral currents (FCNCs) which are strongly constrained  from low-energy measurements of flavour observables. The bounds are of order $|(\tilde{C}_{u\phi,d\phi})_{12}| \lesssim 10^{-5}\Lambda^2/v^2$ and $|(\tilde{C}_{u\phi,d\phi})_{13}| \lesssim 10^{-4} \Lambda^2/v^2$ and stem from $\Delta F=2$ transitions~\cite{Blankenburg:2012ex, Harnik:2012pb}. Given that FCNCs need to be suppressed, a popular way of realising this is by imposing minimal flavour violation (MFV)~\cite{DAmbrosio:2002vsn}, where all sources of flavour violation are proportional to the SM Yukawa couplings
\begin{align}
	(C_{u\phi})_{ij} &= \bar{a}_u \, y_{ij}^u + \bar{b}_u (y_u y_u^{\dagger}) (y_u)_{ij}+ \bar{c}_u (y_d y_d^{\dagger})( y_u)_{ij} + ... \,, \nonumber
	\\ (C_{d\phi})_{ij} &= \bar{a}_d \, y^d_{ij} + \bar{b}_d (y_d y_d^{\dagger}) (y_d)_{ij}+ \bar{c}_d ( y_u y_u^{\dagger}) ( y_d)_{ij} + ...\;.
	\label{eq:mfv}
\end{align}
Here $\bar{a}, \bar{b}$ and $\bar{c}$ are generic flavour universal $\mathcal{O}(1)$ coefficients\\ The assumption of MFV introduces a strong hierarchy amongst the Higgs couplings to quarks, due to the proportionality of the Wilson coefficients to the Yukawa couplings. Since we want to explore rather large modifications of the light-quark Yukawa couplings, in MFV models very low values of the NP scale $\Lambda$ and/or large Wilson coefficients need to be assumed, rendering the validity of the EFT questionable. Furthermore, this would potentially generate conflict with measurements of the third generation couplings to the Higgs boson. Hence, we refrain from assuming MFV and instead assume \textit{flavour alignment}. We will discuss in the next section how this can be concretely realized. Moreover, we choose setting $\Lambda= 1\,\mathrm{TeV}$ throughout the reminder of this paper, staying well within the SMEFT validity region and in order to simplify the presentation of the results.


%%%%%%%%%%%%%%%%%%%%%%%%
\section{Models of flavour alignment and large light-quark Yukawa couplings}
\label{sec:Model}
%%%%%%%%%%%%%%%%%%%%%%%%

A systematic generalisation of flavour alignment is provided by aligned flavour violation (AFV) \cite{Egana-Ugrinovic:2018znw, Egana-Ugrinovic:2019dqu}.  In order to introduce more flavour violation in than~(MFV), AFV introduces more spurions, with the constraint that these spurions are invariant under the~$U(1)^5$ transfirmations mentioned above. This leaves only the CKM matrix transforming non-trivially under the $U(1)^5$. In AFV, it is possible to write these spurions, for example, the extra couplings to up-type and down-type quarks, $k_u$ and $k_d$ respectively, as an expansion in the CKM matrix $V_{CKM}$, known as the alignment expansion 
\begin{align}
	k_u &=  \mathcal V_{L}^u    \left( K_{0,u}+ K_{1,u} V^*_{CKM} K_{2,u} V^T_{CKM} K_{3,u} + \mathcal O(V^4_{CKM})\right)  (\mathcal{V}^u_{R})^\dagger ,  \\
	(k_d)^\dagger&=  \mathcal V_{L}^d  \left( K_{0,d}+ K_{1,d} V^T_{CKM} K_{2,d} V^*_{CKM} K_{3,d} + \mathcal O(V^4_{CKM})\right) \mathcal  (\mathcal{V}^d_{R})^\dagger,
	\label{eqK}
\end{align}
where $K_{a,u}$ and $K_{a,d}$ are complex $3\times3$ diagonal matrices, that are arbitrarily flavour invariant, and the transformation matrices are similar to the ones appearing in eq~\eqref{eq_defV}, they are a generalisation to the SM bi-unitary transformations. The AFV condition necessitates that the alignment coefficients~$K_{a,q}$ to be diagonal, such that the expansion is invariant under the~$U(1)^5$ transformation.  We have omitted generation indices here for readability.\\ applying AFV to the SMEFT case is rather straightforward, in a generic flavour basis we have. 
\begin{equation}
	(k_q)_{ij} = \frac{(C_{q\phi})_{ij}}{\Lambda^2}.
\end{equation}
%%%%%
This formalism is stable under renormalisation group~(RGE) evolution as only the matrices~$K_{i,d}$ will contribute to the RGE and flavour alignment is maintained.
%In general, the non-hierarchical Yukawa corrections from aligned flavour theories can appear through loop-induced FCNC processes not protected by the GIM mechanism thus setting further, and in most cases, stronger flavour constraints on the allowed parameter space of the model. 
%Currently, the improved experimental measurement of $D$-meson mixing \cite{LHCb:2021ykz} or decay may still allow for sizeable enhancement of the down-type light-quark Yukawa given the large non-perturbative theory uncertainty dominating $\Delta F=2$ transitions in the $D$ meson system. \rg{I do not understand why we discuss now $D$-meson mixing if we say that everything remains flavour alignd.} Measurement from $K$- and $B$-meson mixing and decays with improved SM prediction would put strong bounds on allowed enhancement to up-type quark Yukawa couplings, which is not evaluated in Refs.~\cite{Egana-Ugrinovic:2018znw, Egana-Ugrinovic:2019dqu}.
%, and \cite{Bar-Shalom:2018rjs} avoids such constraints by allowing only scalar mixing of the VLQs with the SM. Even though the bounds highly depend on the structure of the detailed setup of the model or its further UV completion, it certainly calls for additional symmetry/suppression to naturally avoid loop-level FCNC bounds. 
%\textcolor{red}{RG: I took over this part from what Zhuoni wrote but I am not sure about it. Which kind of diagrams would contribute to the loop level FCNCs? Are these the ones with the $W$ bosons? Why they should be generically be bigger as in the SM? When drawing a FV diagram with a Higgs you can see easily by chiral analysis that there is a mass suppression.}

%%%%%%%%%%%%%%%%%%%%%%%%%%%
\subsection{Model realizations}
%%%%%%%%%%%%%%%%%%%%%%%%%%%

It should be noted that from a UV perspective there is no well-motivated symmetry argument for the realization of AFV, given the fact that the $U(1)^5$ symmetry is only an auxiliary group used when redefining the quark mass eigenstates. Concrete realisations in UV models are rather fine-tuned and other mechanisms might be required for the realisation of flavour alignment.

Flavour alignment can be realized in various models, for instance SUSY ~\cite{Nir:1993mx,Leurer:1993gy}, two or multi-Higgs doublet models~\cite{Branco:1996bq,Penuelas:2017ikk} and models with vector-like quarks (VLQ)~\cite{Bar-Shalom:2018rjs}.  In the latter,
% Another concrete construction for large Yukawa coupling modifications involving VLQ has been attempted in 
% In the case of multi-Higgs doublet models, the quark-sector Yukawa interaction Lagrangian takes the form
% \begin{equation}
	% -\mathcal L = \sum_a \bar Q_L \left[ \Gamma_a \phi_a d_R + \Delta_a \tilde \phi_a u_R\right]+ \mathrm{h.c.},
	% \end{equation}
% with the summation over the number of Higgs doublets, and $\Phi_1$ is the scalar doublet identified with the "SM" Higgs. The flavour alignment condition manifest in the relation amongst the Yukawa matrices
% \begin{eqnarray}
	% 	\Gamma_a =& e^{-i \theta_a} \xi_a^{d} \hspace{1 cm} \Gamma_1 \Delta_a=& e^{i \theta_a} \xi_a^{u}\Delta_1, \nonumber \\
	% & \xi_1 =1 \,\,\, \xi_{a \neq 1} \in \mathbb{C} &
	% \end{eqnarray}
% It is possible then to obtain large Yukawa enhancement for the first generation quarks by setting large values for $\xi_{a \neq 1} $ for up  and/or down quarks. And tuning -to some degree- the alignment amongst the scalar doublets.\\
FCNCs are avoided by imposing horizontal flavour symmetries leading to AFV. 
The mixing between the SM quarks and the VLQ's $Q\sim (\irrep{3}_{SU(3)_C},\irrep{2}_{SU(2)_L},1/6_Y)$, $U\sim(\irrep{3},\irrep{1},2/3)$ and $D\sim(\irrep{3},\irrep{1},-1/3)$ is given by the Lagrangian 
\begin{equation}
	\begin{split}
		\mathcal{L}=&-\lambda_{Qu} \bar{Q}_L \tilde{\phi} u_R -\lambda_{Qd} \bar{Q}_L \phi d_R-\lambda_{Uq} \bar{q}_L \tilde{\phi} U_R -\lambda_{Dq} \bar{q}_L \phi D_R \\ &-\lambda_{QD}\bar{Q}_L \phi D_R-\lambda_{UQ} \bar{Q}_L \tilde{\phi} U_R+h.c.
	\end{split}
\end{equation}
The matrices~$\lambda$ are the new spurions in this mode, and they do not need to be diagonal, but they are, by virtue of a horizontal symmetry and particular charge assignment, can be made to obey the AFV assumptions.  If all the new VLQ's have the same mass scale~$M$, we could write the enhancement of the light quarks-Higgs coupling~$\delta g_{h q\bar{q}}$ in terms of these matrices
\begin{equation}
	\delta g_{h\bar{u}u} \approx \frac{v^2}{m_{Q}^2} (\lambda_{Uq} \lambda_{UQ} \lambda_{Qu}) \quad \text{ and }   \quad  \delta g_{h\bar{d}d} \approx \frac{v^2}{M^2} (\lambda_{Dq} \lambda_{DQ} \lambda_{Qd}),
\end{equation}
here, the flavour indices are also dropped.
While the exact proposal of~\cite{Bar-Shalom:2018rjs} foresees Yukawa couplings of the first and second generation quarks up to the value of the bottom quark Yukawa coupling, it requires masses of the VLQs of around 1.5 TeV. For less significant enhancements, the scale of the VLQs could reach $>2$ TeV and hence be well within the EFT limit and in accordance with bounds from direct searches of VLQs. The VLQs would also modify Higgs production in gluon fusion and the loop-induced Higgs decays. The contributions would scale like
\begin{equation}
	\frac{\alpha_s}{\pi}\frac{\lambda_{QU}}{m_Q^2}  \phi^{\dagger}\phi G^{\mu\nu} G_{\mu\nu} 
\end{equation}
which with $\mathcal{O}(1)$ $\lambda_{QU}$ is suppressed strongly compared to the top quark contributions.
The VLQs can in principle contribute to flavour observables through loop contributions, where they would couple with the charged currents. The mixing can be fine-tuned such that the loop contributions of the VLQs in flavour observables can be strongly suppressed.


Another concrete realisation of models with large light quark Yukawa couplings has been provided by the framework of spontaneous flavour violation~(SFV)~\cite{Egana-Ugrinovic:2018znw}, where the tuning necessary in general AFV is avoided by promoting the flavour violating spurions to the wavefunctions of the quark~\cite{Egana-Ugrinovic:2018znw}. The wavefunction renormalisation constants come from tree-level diagrams involving interaction between the SM quarks, new set of VLQs and scalars. Unlike general AFV, large deviations from their SM values are possible only for either up-type or down-type Yukawa couplings but not both in the SFV framework. Also, it requires the introduction of a discrete symmetry in order to prevent the VLQ from interacting with the SM degrees of freedom directly. A UV-complete model with SFV has been proposed in Refs.~\cite{Egana-Ugrinovic:2019dqu,Egana-Ugrinovic:2021uew} based on a two-Higgs doublet model.  Deviations in light quark Yukawa coupling can be achieved by having a second Higgs doublet coupling to fermions via flavour diagonal matrices $K_{0,q}$ ($q=u,d$) as defined in~\eqref{eqK}. In particular in order to have only the first/second generation deviating from its SM value the corresponding diagonal element in $K_{0,q}$ is supposed to be non-zero. These matrices are then not proportional to the SM Yukawa couplings. SFV  is realized either only in the down or only in the up sector, assuming that the flavour mixing that generates the CKM matrix correspondingly stems from the other sector.
Assuming that the mass eigenstates of the two Higgs neutral states are
\begin{align}
	h=\sin (\beta-\alpha) h_1+ \cos(\beta-\alpha) h_2\,\\
	H=-\cos(\beta-\alpha) h_1 +\sin(\beta-\alpha) h_2\,
\end{align}
where $h_1$ and $h_2$ are the CP-even neutral interaction eigenstates, the large Yukawa couplings of $h_2$ appear in $g_{h\bar{q}_iq_i}$, the Yukawa coupling of the SM-like Higgs boson, via the mixing of $h_1$ and $h_2$ with the mixing angle $\beta-\alpha$.  Working in the Higgs basis in which $h_1$ takes a vacuum expectation value, the couplings of the Higgs boson to quarks $g_{hq_i\bar{q}_i}$ then become
\begin{equation}
	g_{h\bar{q_i}q_i}= \frac{m_{q_i}}{v} \sin(\beta-\alpha) + (K_{q})_{ii}\cos{(\beta-\alpha)}\,.  \label{eq:ghqq2HDM}  
\end{equation}
where $(K_{q})_{ii}$ is the $i$th matrix element of the matrices $K_{0,d}$ or $K_{0,u}$ appearing in eq~\eqref{eqK}, and the $q$ indicates either up or down-type quarks. Note that the new matrix~$K_{q}$ is simultaneously diagonalisable with the SM Yukawa.
Clearly, the deviation of the Higgs to quark couplings become more pronounced away from the alignment limit $\cos(\beta-\alpha)\to 0$. This automatically leads to a deviation in the Higgs couplings, for instance, to vector bosons proportional to $\cos(\beta-\alpha) $ but not potentially enhanced by large $K_{q}$'s as in case of the Higgs couplings to light quarks. We note that while there is freedom to chose values for $K_q$'s in the diagonal, a potential $\mathcal{O}(1)$ choice would lead to much larger modification factors for first and second generation as for the third generation where the first term in eq.~\eqref{eq:ghqq2HDM} would dominate.
%\rg{One could do a discussion here on the implication of the enhanced light quark Yukawa couplings: small change in gluon fusion, new production channel and suppressed BRs but that would potentially be the same as the single/double Higgs discussion.}
Large $K_q$'s lead also to large couplings of the second Higgs doublet with light quarks 
\begin{equation}
	g_{H q_i \bar{q_i}} \approx (K_{q})_{ii}\sin{(\beta-\alpha)}  \,.  
	\label{Hqq}
\end{equation}
We note that if we want to achieve large deviations to light quarks, the alignment parameter given by
\begin{equation}
	\cos{(\beta-\alpha)}=-\lambda_6 \frac{v^2}{m_H^2}\left[1+\mathcal{O}(v^4/m_H^4) \right],
	\label{lambda6}
\end{equation}
where $m_H$ denotes the heavy Higgs mass and $\lambda_6$ is the 2HDM potential parameter of the operator $\phi_1^{\dagger}\phi_1 \phi_1^{\dagger}\phi_2$, cannot be too small. This implies that the heavy Higgs boson cannot be too heavy if requiring perturbative $\lambda_6$. Given this, the model is constrained from heavy Higgs boson searches as well as meson mixing due to diagrams involving heavy charged Higgs bosons.